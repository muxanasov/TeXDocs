\section{Evaluation}\label{sec:eval}

We implemented three representative applications in scenarios of wildlife
monitoring, smart-home, and adaptive protocol selection. For each we compare
the nesC implementation against its \conesc counterpart.

We estimated complexity by using such metrics as the number of lines of code
(LOC), the number of variables declared and functions defined~\cite{Pressman01}.
Despite our results show increase of the number of LOC and declared functions --
17\% and 50\% correspondingly -- the average number of LOC and functions
per-module is decreased by 50\% and 44\%. The modules in \conesc are smaller
and more decoupled, thus, developing, debugging and supporting processes are
much easier. \conesc-written applicaton is more than 3 times smaller than
generated code, what makes \conesc an effective tool for COP.

The memory overhead is less than 2,5\% for binary size and less than
4,5\% for RAM. The average CPU overhead for layered function calls oscillates
from 2 to 5 CPU cycles depending on application, which is negligible in terms of
energy consumption, since the simplest operation in TinyOS -- turn on/off LEDs
-- consumes 8 CPU cycles. The context transitions overhead is bigger -- from 10
to 25 CPU cycles -- but remains in the same order of magnitude.
