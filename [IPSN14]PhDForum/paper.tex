\documentclass{sig-alternate}

\usepackage{xspace}
\usepackage{graphicx}
% \DeclareGraphicsExtensions{.pdf,.jpeg,.png}
% \usepackage{url}
%\usepackage[procnames]{listings}
\usepackage{paralist}
\usepackage{listings}
\usepackage{xcolor}

% No space between bibliography items:
\let\oldthebibliography=\thebibliography
  \let\endoldthebibliography=\endthebibliography
  \renewenvironment{thebibliography}[1]{%
    \begin{oldthebibliography}{#1}%
      \setlength{\parskip}{0ex}%
      \setlength{\itemsep}{0ex}%
  }%
  {%
    \end{oldthebibliography}%
  }

\renewcommand{\ttdefault}{pcr} % to fix non-bold key-words in listings
\lstdefinelanguage{conesc}
{
 morekeywords={components, triggers, is, error, default, context, group, layered, command, implementation, contexts, event, call, void, uses, transitions, iff, activate, interface},
 morecomment=[s]{/*}{*/},
 sensitive=false,
 escapeinside={*}{*}
 \setcounter{lstNoteCounter}{0}
}

\lstset{
  basicstyle=\scriptsize\ttfamily,
  keywordstyle=\bfseries,
  showstringspaces=false,
  emphstyle=\bfseries,
}

\definecolor{Lemon}{HTML}{FFFACD}
\lstdefinestyle{conescframe}{
  language=conesc,
  frame=lines,%single,
  xleftmargin=4mm,
  framexleftmargin=4mm,
  numbers=left,
  numberstyle=\scriptsize,
  numbersep=4pt,
  backgroundcolor=\color{Lemon}
}

% -begin- //nice in-code glyphs routine
\usepackage{tikz}
% for the pretty dark-circle-enclosed numbers
\newcommand*\circled[2]{\tikz[baseline=(char.base)]{
            \node[shape=circle,fill,inner sep=0.1pt, text=white,font=#2] (char) {#1};}}

\newcounter{lstNoteCounter}
\newcommand{\lstref}[1]{\circled{\ref{#1}}{\normalsize}}
\newcommand{\lstnote}[1] {
  \label{#1}\hbox to -8.5pt{\llap{{\circled{\ref{#1}}{\scriptsize\rmfamily}}\hskip 0.98em}}
}
% -end- //nice in-code glyphs routine

% correct bad hyphenation here
\newcommand{\lm}[1]{\footnote{{\bf Luca: #1}}}
\newcommand{\ma}[1]{\footnote{{\bf Mikhail: #1}}}

\newcommand{\conesc}{{\textsc{ConesC}}\xspace}
\newcommand{\code}[1]{{\bfseries\texttt{#1}}}

\newcommand{\fakepar}[1]{\vspace{0.4mm}\noindent\textbf{#1.}}
\newcommand{\wrt}{w.r.t.\ }

%\linespread{0.97}

\begin{document}

\title{A Context-oriented Approach to Self-adaptive Software\\ in Resource-constrained Cyberphysical Systems}

% Author names and affiliations
% use a multiple column layout for up to two different
% affiliations

\numberofauthors{3} 
\author{
% You can go ahead and credit any number of authors here,
% e.g. one 'row of three' or two rows (consisting of one row of three
% and a second row of one, two or three).
%
% The command \alignauthor (no curly braces needed) should
% precede each author name, affiliation/snail-mail address and
% e-mail address. Additionally, tag each line of
% affiliation/address with \affaddr, and tag the
% e-mail address with \email.
%
% 1st. author
\alignauthor
Mikhail Afanasov \\
       \affaddr{Politecnico di Milano, Italy}\\
       \email{afanasov@elet.polimi.it}
}

% \author{ 
% \alignauthor 
% %
% M.~Afanasov$*$, L. Mottola$*\dagger$ and C.~Ghezzi$*$\\
% %
% \affaddr{$*$Politecnico di Milano (Italy), $\dagger$SICS Swedish ICT} \\
% }

% make the title area
\maketitle

\begin{abstract}
Abstract goes here
\end{abstract}

\section{Introduction}
Here we show a motivation example and emphasise main challenges. (shorten
version from NIER)
spare a pragraph for related work

\section{Design concepts and language support}
Our main concepts are introduced here. Then we show how these concepts are
mapped to the language level. (shorten version from NIER)
divide into two subsections

\section{Evaluation}
We can use evaluation I have done for DCOSS paper. (shorten version)
Future work may include some thoughts about model verification of our approach.

\section{Future work}
Speculate a bit more than in NIER paper.

\section{Related work}
Shorten version of NIER paper.

\section{Biography}
% acknowledgements

% {\small
%\bibliographystyle{plain}      % 
%\bibliography{bibl}   % name your BibTeX data base
% }

\end{document}


