\section{Introduction}
Cyberphysical systems (CPSs) are usually used to gather data form and take
actions on the real world. Because of environment dynamics, CPS software is
required to self-adapt to different situations. It is hard to achieve having
limited resources.

{\bfseries Example.} In wildlife tracking application~\cite{Pasztor10} sensor
nodes are attached to animals to study their movements, social interactions and
health conditions. The fact that the nodes are running on batteries makes the
energy precious resource, which should be used carefully. To this end, such
devices as GPS and high-power radio should be disabled when not needed.

{\bfseries Challenges.} The combination of multiple accepts, the developer has
to take care of, determines the main challenge in implementation of such
software. In our example, the software behavior depends on physical location,
battery level etc., what is even more difficult to implement under resource
limitation.

{\bfseries Related work.} There are many contributions related to self-adaptive
embedded systems~\cite{Zimmerling12,Bourdenas11}, they are, however, very
problem specific. A more general approach includes COP
paradigm~\cite{Hirschfeld08}, which is implemented in several high-level
languages~\cite{Bardram05,Ghezzi10,Kamina11,Salvaneschi12,Sehic11}. Most of them
are not applicable for the embedded systems, because of resource limitations. In
this area CPSs are usually seen as sources of raw data~\cite{Sehic11}.

Our approach targets the applications where complexity arises form the
combination of the environmental dimensions. By borrowing COP concepts and
bringing them down to he low-level software, we aim to provide a programming
support to enable self-adaptive behavior at design-time.
