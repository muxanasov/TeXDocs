\section{Introduction}
Cyberphysical systems (CPSs) are usually used to gather data form and take
actions on the real world. Because of environment dynamics, CPS software is
required to self-adapt to different situations having limited resources.

{\bfseries Example.} In wildlife tracking application~\cite{pasztor10:selective} sensor
nodes are attached to animals to study their movements, social interactions and
health conditions. Nodes are running on batteries, what makes the
energy precious resource. To save it, such
devices as GPS and radio should be disabled when not needed.

{\bfseries Challenges.} The combination of multiple aspects, the developer has
to take care of, determines the complexity of such software. In our example, the
software must behave depending on physical location, battery level etc. under resource limitation.

{\bfseries Related work.} There are many contributions related to self-adaptive
embedded systems~\cite{Zimmerling12,Bourdenas11}, they are, however, very
problem specific. A more general approach -- context-oriented programming (COP)~\cite{Hirschfeld08} -- is implemented in several high-level
languages~\cite{Bardram05,Ghezzi10,Kamina11,Salvaneschi12,Sehic11}, most of them
are not applicable for the embedded systems, because of resource limitations.

By borrowing COP concepts and bringing them down to he low-level software, we
aim to provide a design-time support to enable self-adaptive behavior.
