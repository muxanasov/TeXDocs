\section{Introduction}

Cyberphysical systems (CPSs) are usually used to ga-ther data form and take
actions on the real world. CPS software is required to self-adapt to
environmental dynamics under limited resources.

In wildlife tracking application~\cite{pasztor10:selective} sensor nodes are
attached to animals to study their movements, social interactions, and health
conditions. Nodes are running on batteries, which makes the energy a precious
resource. To save it, such devices as GPS and radio should be disabled when not
needed.

The combination of multiple aspects the developer has to take care of determines
the complexity of such software. In our example, the software must behave
depending on physical location, battery level etc.

Many works exist in the area of self-adaptive embedded
systems~\cite{Zimmerling12,Bourdenas11}, which are very problem-specific. A more
general approach -- context-oriented programming (COP)~\cite{Hirschfeld08} -- is
implemented in several high-level
languages~\cite{Ghezzi10,Salvaneschi12,Sehic11}. Most of them are not applicable
to the embedded systems, because of resource limitations.

By borrowing COP concepts and bringing them down to low-level software, we
provide a design-time and programming support to enable self-adaptive
behavior.
