\section{Design Patterns}

\conesc, as we mentioned before, makes the components of application more
decoupled and reusable. To show that, we provide three possible patters --
general solutions to a commonly occurring problems. The patterns are formalized
best practices, the programmer should follow, and can not be transformed into
source or machine code directly without proper implementation.

\subsection{Behavior Control Pattern}

This pattern directly arise from the motivation of our work. Whenever the
application should perform some specific actions repeatedly, according to the
environmental conditions, the developer should use \emph{Behavior Control
Pattern}. The example is in our scenario, when the application adapts to the
reach-ability of the base-station.

Context group in that case includes the functionality, which needs to adapt.
Each context implies an implementation of a corresponding functionality. To make
the pattern complete, we also use a separate control module, which is
responsible only for the activating suitable context in the given context group.

\subsection{Content Provider Pattern}

Whenever the internal logic of the application implies the component which
generates data, it might be useful to use this pattern. Here, a content provider
is the source of the data, which is dependent on the environmental conditions.
As, for example, in our scenario the beacon contains the information -- the
health conditions -- which is dependent on the body temperature.

Differently from the Behavior Control Pattern, the functionality of the context
provider does not imply the control of the execution. The component
architecture, however, is the same.

\subsection{Context Initialization Pattern}

Since the \emph{context} is a separated and isolated module, it should be
initialized whenever activated. One way to do that is to provide a \code{command
init()} function, but it leads to coupling type \emph{Control}, when one module
controls the execution of another by passing the information, how to execute. To
avoid the coupling, one should use event \code{activated()} to perform all
necessary initializations.

