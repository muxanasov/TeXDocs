\section{Introduction}

Cyberphysical systems (CPSs) place a computing and communication core
in the environment to gather data from, and possibly take actions on
the real world. Because of the intimate interactions between the
``world'' and the ``machine''~\cite{Jackson:1995:WM:225014.225041},
CPS software is eminently required to self-adapt against the many and
unpredictable environment dynamics. This is difficult to achieve in
general~\cite{cheng:adaptive}, and even more so when developers are to
battle against the resource limitations of many existing CPS
platforms.

\fakepar{Example} Consider a wireless sensor network application for
wildlife monitoring~\cite{pasztor10:selective}.  Sensor nodes are
embedded in collars attached to animals, e.g., badgers, to study their
social interactions. The nodes are equipped with sensors to track an
animal's movement, e.g., using GPS and accelerometers, and to detect
its health conditions, e.g., based on body temperature.  A low-power
short-range radio allows the nodes to discover each other based on
periodic radio beaconing.  A node logs the radio contacts to track an
animal's encounters with other animals. The radio is also used to
off-load the contact traces when in reach of a fixed base-station.

The nodes run on batteries, making energy a precious resource that
developers may need to trade against the system's functionality,
depending on the situation. For example, sensor sampling consumes
non-negligible energy, especially for devices such as the
GPS. Depending on the desired granularity and on the difference
between consecutive GPS readings---the latter taken as indication of
the pace of movement---developers may tune the GPS sampling frequency
accordingly. The contact traces can be sent directly to the
base-station whenever in reach, but they need to be stored locally on
a node otherwise.  When the battery is running low, developers may
turn the GPS sensor off to make the node survive until the next
encounter with a base-station, not to lose the collected contact
traces.

\fakepar{Challenges and new ideas} Taking into explicit account every
possible situation in the design of CPS software is a
challenge. Critically, \emph{multiple combined aspects} concurrently
determine how the software should adapt its operation, e.g., battery
levels and physical locations in our example application. Although the
existing literature already investigates similar
problems~\cite{cheng:adaptive}, a principled approach at tackling them
in the design and implementation of CPS software for extremely
resource-constrained platforms is lacking. As a result, existing
implementations are very entangled, and thus difficult to debug and to
maintain~\cite{Picco:2010:SEW:1882362.1882421}.

We address this issue by presenting context-oriented design concepts
and a corresponding programming model expressly conceived for
resource-constrained CPS platforms. To this end, we define in
Section~\ref{sec:design} a specific notion of \emph{context} and
\emph{context group}, useful to conceptually organize the different
situations the system may find itself in, and their combinations. This
provides support during the design phases. We reflect these notions in
a custom \emph{programming model}, described in
Section~\ref{sec:conesc}, which brings concepts of context-oriented
programming~\cite{Hirschfeld08} in traditional low-level CPS programming
languages~\cite{gay03nesc}. Notably, these are characterized by severe restrictions,
e.g., the inability to create run-time instances of components.

Section~\ref{sec:eval} illustrates early results indicating that our
approach results in better structured implementations, where
components are more decoupled and thus simpler to test and to
maintain. We further demonstrate that accounting for changing
requirements is likely easier in our approach. These results come at a
negligible increase in resource consumption, which does not impact the
feasibility of our approach on the target platforms. For example, we
observe a mere 3\% increase in program memory, whereas the energy
overhead is negligible.

We conclude the paper with a brief account of related efforts in
Section~\ref{sec:related}, and with brief concluding remarks and
directions for future work in Section~\ref{sec:ending}.

% \hrule

% \noindent Nuggets to play as motivation:
% \begin{itemize}
% \item CPSs intimately tied to the environment;
% \item WSNs are an extreme case of resource constraint;
% \item applications need to evolve because i)  requirements change,
%   ii) environment changes, iii) nodes possibly move;
% \item software development rather primitive, no established methodologies, low-level languages;
% \item resource constraints make things worse: no dynamic memory
%   allocation, fixed component bindings, ...
% \end{itemize}

% Follows a description of the approach: use a notion of context (and
% context groups) at the design stage together with dedicated COP
% support during the implementation. We are the first to bring a similar
% approach down to extremely resource-constrained platforms.

% Throughout the paper, use wildlife monitoring application as running
% example.



%%% Local Variables: 
%%% mode: latex
%%% TeX-master: "paper"
%%% End: 
