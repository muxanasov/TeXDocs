\section{Introduction}

Cyberphysical systems (CPSs) place a computing and communication core
in the environment to gather data from, and possibly take actions on
the real world. Because of the intimate interactions between the
system and the physical world it is immersed in, CPS software is
eminently required to self-adapt against the many and unpredictable
environment dynamics. This is difficult to achieve in
general~\cite{cheng:adaptive}, and even more so whenever developers
are to battle against the resource limitations of many existing CPS
platforms.

\fakepar{Example} Consider a wireless sensor network application for
wildlife monitoring~\cite{pasztor10:selective}.  Sensor nodes are
embedded in collars attached to animals, e.g., badgers, to study their
social interactions. The nodes are equipped with sensors to track an
animal's movement, e.g., using GPS and accelerometers, and to detect
its health conditions, e.g., based on body temperature.  A low-power
short-range radio allows the nodes to discover each other through
periodic radio beaconing.  A node logs the radio contacts to track
an animal's encounters with other animals. The radio is also used to
off-load the contact traces when in reach of a fixed base-station.

The nodes run on batteries, making energy a precious resource that
developers may need to trade against the system's functionality,
depending on the situation. For example, sensor sampling consumes
non-negligible energy, especially for devices such as the
GPS. Depending on the desired granularity and on the difference
between consecutive GPS readings---the latter taken as indication of
the pace of movement---developers may tune the GPS sampling frequency
accordingly. The contact traces can be sent directly to the
base-station whenever in reach, but they need to be stored locally on
a node otherwise.  When the battery is running low, developers may
turn the GPS sensor off to make the node survive until the next
encounter with a base-station, not to lose the collected contact
traces.

\fakepar{Contribution and road-map} Taking into explicit account every
possible enviroment situation in the design of CPS software is a
challenge. Crucially, \emph{multiple combined aspects} concurrently
determine how the software should adapt its operation, e.g., battery
levels and physical locations in our example application. Although the
existing literature already investigates similar
problems~\cite{cheng:adaptive}, a principled approach at tackling
these issues in the design and implementation of CPS software for
extremely resource-constrained platforms is largely missing. The
platforms' characteristics, such as battery-powered operation and
limited memory budgets, make this a challenge.

Existing component-based frameworks for sensor
networks~\cite{mottola10:survey}, for example, employ component-based
programming during development, but sacrifice this notion at run-time
to mitigate resource limitations. In the nesC
language~\cite{gay03nesc}, components are in-lined during compilation
to enable whole-program analysis, meant to aggressively reduce the
size of the program binary to fit the limited memory. This prevents
runtime creation of component instances and dynamic component binding,
which may help implement self-adaptation functionality by employing
different components according to the situation at stake. Programmers
often circumvent these limitations by ``emulating'' these
functionality with hand-written specialized
code~\cite{mottola10:survey}. As a result, implementations become
entangled, and are thus difficult to maintain and
evolve~\cite{Picco:2010:SEW:1882362.1882421}.

We address this issue by presenting context-oriented design concepts
and a corresponding programming model expressly conceived for
resource-constrained CPS platforms. To this end, we define in
Section~\ref{sec:design} a specific notion of \emph{context} and
\emph{context group}, useful to conceptually organize the different
situations the system may find itself in, and their combinations. This
provides support during the design phases. We reflect these notions in
a custom \emph{programming model}, described in
Section~\ref{sec:conesc}, which brings concepts of context-oriented
programming~\cite{Hirschfeld08} in existing component-based frameworks
for resource-constrained CPSs~\cite{gay03nesc}. % Notably, these are
% characterized by severe restrictions, e.g., the inability to create
% run-time instances of components because of the few KBytes of data and
% program memory typically available.

Section~\ref{sec:eval} illustrates early results indicating that our
approach may result in better structured implementations, where
components are increasingly decoupled. We further demonstrate that accounting
for changing requirements is likely easier in our approach. These
results come at a negligible increase in resource consumption, which
does not impact the feasibility of our approach on the target
platforms. For example, we observe a mere 3\% increase in program
memory, whereas the energy overhead is negligible.

We conclude the paper by discussing in Section~\ref{sec:patterns}
recurring design and programming patterns that we already observe
emerging in our experience, and by briefly surveying related efforts
in Section~\ref{sec:related}. Section~\ref{sec:ending} ends the paper
with brief concluding remarks and an agenda for future work.

% \hrule

% \noindent Nuggets to play as motivation:
% \begin{itemize}
% \item CPSs intimately tied to the environment;
% \item WSNs are an extreme case of resource constraint;
% \item applications need to evolve because i)  requirements change,
%   ii) environment changes, iii) nodes possibly move;
% \item software development rather primitive, no established methodologies, low-level languages;
% \item resource constraints make things worse: no dynamic memory
%   allocation, fixed component bindings, ...
% \end{itemize}

% Follows a description of the approach: use a notion of context (and
% context groups) at the design stage together with dedicated COP
% support during the implementation. We are the first to bring a similar
% approach down to extremely resource-constrained platforms.

% Throughout the paper, use wildlife monitoring application as running
% example.



%%% Local Variables: 
%%% mode: latex
%%% TeX-master: "paper"
%%% End: 
