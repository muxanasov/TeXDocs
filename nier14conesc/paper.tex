\documentclass[10pt, conference, compsocconf]{IEEEtran}

\usepackage{xspace}
\usepackage[pdftex]{graphicx}
\DeclareGraphicsExtensions{.pdf,.jpeg,.png}
\usepackage{url}
\usepackage[procnames]{listings}
\usepackage{paralist}

% correct bad hyphenation here
\newcommand{\lm}[1]{\footnote{{\bf Luca: #1}}}
\newcommand{\ma}[1]{\footnote{{\bf Mikhail: #1}}}

\newcommand{\fakepar}[1]{\vspace{1mm}\noindent\textbf{#1.}}
\newcommand{\wrt}{w.r.t.\ }

%\linespread{0.97}


\begin{document}

\title{A Disciplined Approach to Self-adaptive Software\\ in Resource-constrained Cyberphysical Systems}

% Author names and affiliations
% use a multiple column layout for up to two different
% affiliations

\author{}

% \author{\IEEEauthorblockN{Authors Name/s per 1st Affiliation (Author)}
% \IEEEauthorblockA{line 1 (of Affiliation): dept. name of organization\\
% line 2: name of organization, acronyms acceptable\\
% line 3: City, Country\\
% line 4: Email: name@xyz.com}
% \and
% \IEEEauthorblockN{Authors Name/s per 2nd Affiliation (Author)}
% \IEEEauthorblockA{line 1 (of Affiliation): dept. name of organization\\
% line 2: name of organization, acronyms acceptable\\
% line 3: City, Country\\
% line 4: Email: name@xyz.com}
% }

% make the title area
\maketitle

\begin{abstract}
We present our work towards self-adaptive software for extremely
resource-constrained cyberphysical systems \\(CPSs). The software for the latter
generally needs to adapt to unpredictable environmental dynamics under limited
resources. Our approach provides design concepts and language support for
developing such software. To this end, we bring a notion of context-oriented
design and programming down to embedded platforms, while the system overhead is
negligible.
\end{abstract}

%%% Local Variables: 
%%% mode: latex
%%% TeX-master: "paper"
%%% End: 


\section{Introduction}

\noindent Nuggets to play as motivation:
\begin{itemize}
\item CPSs intimately tied to the environment;
\item WSNs are an extreme case of resource constraint;
\item applications need to evolve because i)  requirements change,
  ii) environment changes, iii) nodes possibly move;
\item software development rather primitive, no established methodologies, low-level languages;
\item resource constraints make things worse: no dynamic memory
  allocation, fixed component bindings, ...
\end{itemize}

Follows a description of the approach: use a notion of context (and
context groups) at the design stage together with dedicated COP
support during the implementation. We are the first to bring a similar
approach down to extremely resource-constrained platforms.

Throughout the paper, use wildlife monitoring application as running
example.

\section{Design Concepts}

Definition of context, context group, transitions among contexts, and
related constraints (towards an error context, or preventing a
transition).

\section{Programming Support}

This is going to be a primer on ConesC, with a mapping from the
concepts in the previous section to specific language constructs.

\section{Preliminary Evaluation}
\noident Three key aspects:
\begin{itemize}
\item qualitatively compare ConesC and nesC implementations along
  coupling, dependencies, programmer-visible program states, ...
\item demonstrates that changing the application (for example, adding
  a new context) is easier in ConesC than nesC;
\item shows a few figures on run-time overhead to demonstrate the
  approach is feasible.
\end{itemize}

\section{Conclusion}

This is the end, pointers to ongoing and future work.

% \begin{IEEEkeywords}
% TBW
% \end{IEEEkeywords}

\IEEEpeerreviewmaketitle

% \bibliographystyle{IEEEtran}      % 
% \bibliography{ref}   % name your BibTeX data base

\end{document}


