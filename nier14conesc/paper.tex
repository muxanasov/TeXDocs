\documentclass[10pt, conference, compsocconf]{IEEEtran}

\usepackage{xspace}
\usepackage[pdftex]{graphicx}
\DeclareGraphicsExtensions{.pdf,.jpeg,.png}
\usepackage{url}
\usepackage[procnames]{listings}
\usepackage{paralist}

% correct bad hyphenation here
\newcommand{\lm}[1]{\footnote{{\bf Luca: #1}}}
\newcommand{\ma}[1]{\footnote{{\bf Mikhail: #1}}}

\newcommand{\fakepar}[1]{\vspace{1mm}\noindent\textbf{#1.}}
\newcommand{\wrt}{w.r.t.\ }

%\linespread{0.97}

\begin{document}

\title{A Context-oriented Approach to Self-adaptive Software\\ in Resource-constrained Cyberphysical Systems}

% Author names and affiliations
% use a multiple column layout for up to two different
% affiliations

\author{}

% \author{\IEEEauthorblockN{Authors Name/s per 1st Affiliation (Author)}
% \IEEEauthorblockA{line 1 (of Affiliation): dept. name of organization\\
% line 2: name of organization, acronyms acceptable\\
% line 3: City, Country\\
% line 4: Email: name@xyz.com}
% \and
% \IEEEauthorblockN{Authors Name/s per 2nd Affiliation (Author)}
% \IEEEauthorblockA{line 1 (of Affiliation): dept. name of organization\\
% line 2: name of organization, acronyms acceptable\\
% line 3: City, Country\\
% line 4: Email: name@xyz.com}
% }

% make the title area
\maketitle

\begin{abstract}
  We present a context-oriented approach to develop self-adaptive
  software for extremely resource-constrained cyber-physical systems
  (CPSs). % These are sensing and actuating systems deployed in the
  % physical world whose operation is controlled by a computing and
  % communication core.
  Because of unpredictable environment dynamics, CPS software must be
  designed and implemented to dynamically adapt to widely different
  situations. Our approach provides design concepts and language
  support to achieve this against extreme resource constraints. To
  this end, we bring a notion of context-oriented design and
  programming down to platforms with only a few KBytes of memory and
  currently leveraging rather basic programming environments. Early
  results demonstrate that our approach greatly simplifies the design
  and implementation of adaptive CPS software at the price of a modest
  system overhead. To our knowledge, we are the first to enable
  context-oriented design and programming in similarly-constrained
  platforms.
\end{abstract}


%%% Local Variables: 
%%% mode: latex
%%% TeX-master: "paper"
%%% End: 


\section{Introduction}

Cyberphysical systems (CPSs) gather data from and take actions on the real
world. The environmental dynamics determines the requirement for CPS software to
be adaptable. Moreover, adaptation should occur simultaneously along diferent
dimensions. In wildlife tracking applications~\cite{pasztor10:selective}, for
example, sensor nodes are attached to animals to study their movements, social
interactions, and health conditions. Nodes are running on batteries, which makes
the energy a precious resource. To save it, such devices as GPS and radio should
be disabled when not needed. Orthogonally, the node should send the data to the
base-station, if it is reachable, or save it locally otherwise.

In the absence of design-time support for self-adaptive software, the adaptation
described above would be achieved by ad-hoc
code~\cite{Zimmerling12,Bourdenas11}, that makes the application cumbersome to
design, and difficult to understand, debug, and maintain. A more general
approach -- context-oriented programming (COP)~\cite{Hirschfeld08} -- addresses
these issues by providing a design-time abstractions for adaptations. There are
many implementations of COP in high-level
languages~\cite{Ghezzi10,Salvaneschi12,Sehic11}, most of which are not
applicable to embedded systems, because of resource limitations.

%Many works exist in the area of self-adaptive embedded
%systems~\cite{Zimmerling12,Bourdenas11}, which illustrates a problem-specific
%approach dedicated code. A more general approach -- context-oriented
%programming (COP)~\cite{Hirschfeld08} -- is implemented in several high-level
%languages~\cite{Ghezzi10,Salvaneschi12,Sehic11}. Most of them are not
%applicable to the embedded systems, because of resource limitations.

By borrowing COP concepts and bringing them down to low-level CPS software, we
provide a design-time and programming support to enable self-adaptive behavior.
As a result, we introduce \conesc -- a context-oriented extension of nesC
language -- in Sec.~\ref{sec:concepts}, where we also describe the concepts the
language was build upon. We discuss our preliminary evaluations in
Sec.~\ref{sec:eval}, and open problems in Sec.~\ref{sec:future}.


\section{Design Concepts}
\label{sec:design}

\hrule
Definition of context, context group, transitions among contexts, and
related constraints (towards an error context, or preventing a
transition).
\hrule

\section{Programming Support}
\label{sec:conesc}

\hrule
This is going to be a primer on ConesC, with a mapping from the
concepts in the previous section to specific language constructs.
\hrule

\section{Preliminary Evaluation}
\label{sec:eval}

\hrule
\noindent Three key aspects:
\begin{itemize}
\item qualitatively compare ConesC and nesC implementations along
  coupling, dependencies, programmer-visible program states, ...
\item demonstrates that changing the application (for example, adding
  a new context) is easier in ConesC than nesC;
\item shows a few figures on run-time overhead to demonstrate the
  approach is feasible.
\end{itemize}
\hrule

\section{Related Work}
\label{sec:related}


\section{Future Work and Conclusion}
\label{sec:ending}

\hrule
This is the end, pointers to ongoing and future work.
\hrule

% \begin{IEEEkeywords}
% TBW
% \end{IEEEkeywords}

\IEEEpeerreviewmaketitle

% \bibliographystyle{IEEEtran}      % 
% \bibliography{ref}   % name your BibTeX data base

\end{document}


