\documentclass{sig-alternate}

\usepackage{xspace}
\usepackage{graphicx}
% \DeclareGraphicsExtensions{.pdf,.jpeg,.png}
% \usepackage{url}
%\usepackage[procnames]{listings}
\usepackage{paralist}
\usepackage{listings}
\usepackage{xcolor}

% No space between bibliography items:
\let\oldthebibliography=\thebibliography
  \let\endoldthebibliography=\endthebibliography
  \renewenvironment{thebibliography}[1]{%
    \begin{oldthebibliography}{#1}%
      \setlength{\parskip}{0ex}%
      \setlength{\itemsep}{0ex}%
  }%
  {%
    \end{oldthebibliography}%
  }

\renewcommand{\ttdefault}{pcr} % to fix non-bold key-words in listings
\lstdefinelanguage{conesc}
{
 morekeywords={components, triggers, is, error, default, context, group, layered, command, implementation, contexts, event, call, void, uses, transitions, iff, activate, interface},
 morecomment=[s]{/*}{*/},
 sensitive=false,
 escapeinside={*}{*}
 \setcounter{lstNoteCounter}{0}
}

\lstset{
  basicstyle=\scriptsize\ttfamily,
  keywordstyle=\bfseries,
  showstringspaces=false,
  emphstyle=\bfseries,
}

\definecolor{Lemon}{HTML}{FFFACD}
\lstdefinestyle{conescframe}{
  language=conesc,
  frame=lines,%single,
  xleftmargin=4mm,
  framexleftmargin=4mm,
  numbers=left,
  numberstyle=\scriptsize,
  numbersep=4pt,
  backgroundcolor=\color{Lemon}
}

% -begin- //nice in-code glyphs routine
\usepackage{tikz}
% for the pretty dark-circle-enclosed numbers
\newcommand*\circled[2]{\tikz[baseline=(char.base)]{
            \node[shape=circle,fill,inner sep=0.1pt, text=white,font=#2] (char) {#1};}}

\newcounter{lstNoteCounter}
\newcommand{\lstref}[1]{\circled{\ref{#1}}{\normalsize}}
\newcommand{\lstnote}[1] {
  \label{#1}\hbox to -8.5pt{\llap{{\circled{\ref{#1}}{\scriptsize\rmfamily}}\hskip 0.98em}}
}
% -end- //nice in-code glyphs routine

% correct bad hyphenation here
\newcommand{\lm}[1]{\footnote{{\bf Luca: #1}}}
\newcommand{\ma}[1]{\footnote{{\bf Mikhail: #1}}}

\newcommand{\conesc}{{\textsc{ConesC}}\xspace}
\newcommand{\code}[1]{{\bfseries\texttt{#1}}}

\newcommand{\fakepar}[1]{\vspace{0.4mm}\noindent\textbf{#1.}}
\newcommand{\wrt}{w.r.t.\ }

%\linespread{0.97}

\begin{document}

\title{A Context-oriented Approach to Self-adaptive Software\\ in Resource-constrained Cyberphysical Systems}

% Author names and affiliations
% use a multiple column layout for up to two different
% affiliations

\numberofauthors{3} 
\author{
% You can go ahead and credit any number of authors here,
% e.g. one 'row of three' or two rows (consisting of one row of three
% and a second row of one, two or three).
%
% The command \alignauthor (no curly braces needed) should
% precede each author name, affiliation/snail-mail address and
% e-mail address. Additionally, tag each line of
% affiliation/address with \affaddr, and tag the
% e-mail address with \email.
%
% 1st. author
\alignauthor
Mikhail Afanasov \\
       \affaddr{Politecnico di Milano, Italy}\\
       \email{afanasov@elet.polimi.it}
% 2nd. author
\alignauthor
Luca Mottola \\
       \affaddr{Politecnico di Milano, Italy and SICS Swedish ICT}\\
       \email{luca.mottola@polimi.it}
% 3rd. author
\alignauthor 
Carlo Ghezzi\\
       \affaddr{Politecnico di Milano, Italy}\\
       \email{carlo.ghezzi@polimi.it}
}

% \author{ 
% \alignauthor 
% %
% M.~Afanasov$*$, L. Mottola$*\dagger$ and C.~Ghezzi$*$\\
% %
% \affaddr{$*$Politecnico di Milano (Italy), $\dagger$SICS Swedish ICT} \\
% }

% make the title area
\maketitle

\begin{abstract}
We present our work towards self-adaptive software for extremely
resource-constrained cyberphysical systems \\(CPSs). The software for the latter
generally needs to adapt to unpredictable environmental dynamics under limited
resources. Our approach provides design concepts and language support for
developing such software. To this end, we bring a notion of context-oriented
design and programming down to embedded platforms, while the system overhead is
negligible.
\end{abstract}

%%% Local Variables: 
%%% mode: latex
%%% TeX-master: "paper"
%%% End: 


\section{Introduction}

Cyberphysical systems (CPSs) place a computing and communication core
in the environment to gather data from, and possibly take actions on
the real world. Because of the intimate interactions between the
system and the physical world it is immersed in, CPS software is
eminently required to self-adapt against the many and unpredictable
environment dynamics. This is difficult to achieve in
general~\cite{cheng:adaptive}, and even more so whenever developers
are to battle against the resource limitations of many existing CPS
platforms.

\fakepar{Example} Consider a wireless sensor network application for
wildlife monitoring~\cite{pasztor10:selective}.  Sensor nodes are
embedded in collars attached to animals, e.g., badgers, to study their
social interactions. The nodes are equipped with sensors to track an
animal's movement, e.g., using GPS and accelerometers, and to detect
its health conditions, e.g., based on body temperature.  A low-power
short-range radio allows the nodes to discover each other through
periodic radio beaconing.  A node logs the radio contacts to track
an animal's encounters with other animals. The radio is also used to
off-load the contact traces when in reach of a fixed base-station.

The nodes run on batteries, making energy a precious resource that
developers may need to trade against the system's functionality,
depending on the situation. For example, sensor sampling consumes
non-negligible energy, especially for devices such as the
GPS. Depending on the desired granularity and on the difference
between consecutive GPS readings---the latter taken as indication of
the pace of movement---developers may tune the GPS sampling frequency
accordingly. The contact traces can be sent directly to the
base-station whenever in reach, but they need to be stored locally on
a node otherwise.  When the battery is running low, developers may
turn the GPS sensor off to make the node survive until the next
encounter with a base-station, not to lose the collected contact
traces.

\fakepar{Contribution and road-map} Taking into explicit account every
possible enviroment situation in the design of CPS software is a
challenge. Crucially, \emph{multiple combined aspects} concurrently
determine how the software should adapt its operation, e.g., battery
levels and physical locations in our example application. Although the
existing literature already investigates similar
problems~\cite{cheng:adaptive}, a principled approach at tackling
these issues in the design and implementation of CPS software for
extremely resource-constrained platforms is largely missing. The
platforms' characteristics, such as battery-powered operation and
limited memory budgets, make this a challenge.

Existing component-based frameworks for sensor
networks~\cite{mottola10:survey}, for example, employ component-based
programming during development, but sacrifice this notion at run-time
to mitigate resource limitations. In the nesC
language~\cite{gay03nesc}, components are in-lined during compilation
to enable whole-program analysis, meant to aggressively reduce the
size of the program binary to fit the limited memory. This prevents
runtime creation of component instances and dynamic component binding,
which may help implement self-adaptation functionality by employing
different components according to the situation at stake. Programmers
often circumvent these limitations by ``emulating'' these
functionality with hand-written specialized
code~\cite{mottola10:survey}. As a result, implementations become
entangled, and are thus difficult to maintain and
evolve~\cite{Picco:2010:SEW:1882362.1882421}.

We address this issue by presenting context-oriented design concepts
and a corresponding programming model expressly conceived for
resource-constrained CPS platforms. To this end, we define in
Section~\ref{sec:design} a specific notion of \emph{context} and
\emph{context group}, useful to conceptually organize the different
situations the system may find itself in, and their combinations. This
provides support during the design phases. We reflect these notions in
a custom \emph{programming model}, described in
Section~\ref{sec:conesc}, which brings concepts of context-oriented
programming~\cite{Hirschfeld08} in existing component-based frameworks
for resource-constrained CPSs~\cite{gay03nesc}. % Notably, these are
% characterized by severe restrictions, e.g., the inability to create
% run-time instances of components because of the few KBytes of data and
% program memory typically available.

Section~\ref{sec:eval} illustrates early results indicating that our
approach may result in better structured implementations, where
components are increasingly decoupled. We further demonstrate that accounting
for changing requirements is likely easier in our approach. These
results come at a negligible increase in resource consumption, which
does not impact the feasibility of our approach on the target
platforms. For example, we observe a mere 3\% increase in program
memory, whereas the energy overhead is negligible.

We conclude the paper by discussing in Section~\ref{sec:patterns}
recurring design and programming patterns that we already observe
emerging in our experience, and by briefly surveying related efforts
in Section~\ref{sec:related}. Section~\ref{sec:ending} ends the paper
with brief concluding remarks and an agenda for future work.

% \hrule

% \noindent Nuggets to play as motivation:
% \begin{itemize}
% \item CPSs intimately tied to the environment;
% \item WSNs are an extreme case of resource constraint;
% \item applications need to evolve because i)  requirements change,
%   ii) environment changes, iii) nodes possibly move;
% \item software development rather primitive, no established methodologies, low-level languages;
% \item resource constraints make things worse: no dynamic memory
%   allocation, fixed component bindings, ...
% \end{itemize}

% Follows a description of the approach: use a notion of context (and
% context groups) at the design stage together with dedicated COP
% support during the implementation. We are the first to bring a similar
% approach down to extremely resource-constrained platforms.

% Throughout the paper, use wildlife monitoring application as running
% example.



%%% Local Variables: 
%%% mode: latex
%%% TeX-master: "paper"
%%% End: 

\section{Design Concepts}
\label{sec:design}

Our approach to providing support to designing self-adaptive CPS
software is intentionally fundamental and conceptually simple: CPS
software targeted at resource-constrained devices hardly requires the
design of sophisticated functionality. Nevertheless, the lack of a
principled approach for its design is currently hampering the field at
large~\cite{gp,cpskeynote}. Key in our approach, nevertheless, is the
co-design of design-time support and programming model. We illustrate
the latter in Section~\ref{sec:conesc}.

We define two key design concepts: \emph{i)} individual
\emph{contexts}; and \emph{ii)} \emph{context groups}, along with the
notions necessary to weave these into a complete design.  Contexts
represent the different environmental situations the system may
encounter, and correspond to behavioral variations associated to a
given situation. As the environment surrounding the system mutates,
the software should adapt accordingly by \emph{activating} a suitable
context. Context groups represent collections of individual contexts
sharing common characteristics; e.g., whenever the \emph{same}
high-level functionality must adapt to changes in the surrounding
environment.

\begin{figure}
\begin{center}
\includegraphics[scale=.45]{imgs/wildlifetracking}
\vspace{-2mm}
\caption{Wildlife monitoring application design.}
  \label{fig:design}
\vspace{-4mm}
\end{center}
\end{figure}

Figure~\ref{fig:design} graphically depicts the context-oriented
design of the example wildlife monitoring applications described
earlier. Context groups are defined to describe behavioral variations
corresponding to battery levels, base-station reachability, as well as
an animal's health conditions and activity. Context groups are tagged
as system-level or application-level to discern application-specific
functionality from likely re-usable system services.

The contexts within a group define the behavioral variations at
hand. For example, the software must behave differently depending on
whether the base-station is reachable.  The contexts in a group are
tied with explicit \emph{transitions}, labeled with the conditions
triggering the context change. For example, the system shall
transition from context ``Base-station reachable'' to ``Base-station
unreachable'' whenever no beacons are received from the base-station
within a specific timeout. This entails a node is out of the
base-station range and the software must adapt accordingly.

The specific software adaptation is encapsulated in the individual
contexts. In the latter, it is useful to distinguish \emph{one-time
  operations} executed at the time of entering or existing a context,
from \emph{continuous activities} that occur as long as a context is
active. For example, whenever entering ``Base-station reachable'', the
software shall dump onto the latter the contact logs
accumulated while the base-station was unreachable. Similarly, when
the latter situation, the software must log the contacts locally as
long as ``Base-station unreachable'' persists, as shown in
Figure~\ref{fig:design}.

Required adaptation functionality may span multiple context groups. To
this end, developers can bind context activations across groups and/or
make context transitions subject to the activation of other
contexts. An example of the former is in ``Nest Activity'': when entering --
the coordinates of the nest are static and well known -- ``Base-station
Reachable'' shall also consequently activate, as the latter is usually deployed
near the nest. An example of conditioned context transitions is when activating
``Base-station reachable''. Besides receiving a beacon from the
base-station, which implies radio connectivity, the adaptation process
must check that ``Nest'' in the ``Animal activity'' group is
currently active. Indeed, gaining radio connectivity to the
base-station entails the animal is close to its nest. Should that not be the
case, developers might have not correctly captured how contexts
evolve, potentially indicating a design error.

The concepts we define provide design-time support to reason on the
different situations the software must adapt to, and to identify common
functionality, orthogonal aspects, and mutual constraints. This
ultimately helps separate concerns during the implementation phase, as
we illustrate next.



%%% Local Variables: 
%%% mode: latex
%%% TeX-master: "paper"
%%% End: 

\section{Programming Support}
\label{sec:conesc}

We render the design concepts above in a set of context-oriented
programming (COP)~\cite{Hirschfeld08} constructs feasible within
existing programming environments for resource-constrained CPS
platforms~\cite{mottola10:survey}.  We exemplify our approach based on
nesC~\cite{gay03nesc}, a mainstream sensor network programming
language. However, our approach is not tied to it, and may be readily
translated to other similar programming systems~\cite{mottola10:survey}.

\fakepar{Target language} nesC is an event-driven programming language
for sensor networks, derived from C. Applications are built by
interconnecting \emph{components} that interact by providing or using
\emph{interfaces}. An interface lists one or more functions, tagged as
\emph{commands} or \emph{events}. % Commands are used to execute
% actions, while events are used to collect the results
% asynchronously.
% Interfaces in nesC are bidirectional: data flows both
% ways between components connected through the same
% interface.
Component \emph{configurations} specify the wirings among
components. % Configurations are component themselves, so they can offer
% interfaces and be wired to other components.

nesC exemplifies the limitations dictated by the target platforms, and
hence the reasons why existing COP approaches cannot be directly
ported. Because of the limited memory available, for example,
components cannot be instantiated at run-time. They are rather
in-lined at compile time so to reduce the size of the executable
binary~\cite{gay03nesc}, which would hardly fit in the available
program memory otherwise.  The use of dynamically-allocated memory is
also discouraged: the MCUs provide no memory protection, so bugs in
memory handling may have disastrous effects.

\fakepar{\conesc} Notwithstanding the above constraints, we design a
context-oriented extension to nesC, called \conesc, that incorporates
the design concepts described in Section~\ref{sec:design}. 

At the core of \conesc is a notion of \emph{layered
  function}~\cite{Hirschfeld08}. These are functions whose behavior
depends on the currently active context, and are hence the primary
means to implement the behavioral variations necessary for
self-adaptation. A \emph{context group} in \conesc extends standard
nesC configurations by also specifying the contexts included in the
group and the layered functions that such contexts provide. The
individual contexts extend the nesC component providing the
context-dependent implementations of layered functions.

\begin{figure}[!tb]
\begin{lstlisting}[style=conescframe]
context group BaseStationG {
*\lstnote{cg:layered}* layered command void report(contact_t contact);
}implementation {
*\lstnote{cg:ctx}* contexts Reachable, 
*\lstnote{cg:def}*          Unreachable is default,
*\lstnote{cg:error}*          ErrorC is error;
 // Standard nesC component wirings... }
\end{lstlisting}
\vspace{-5mm}
\caption{Context group in \conesc.}
  \label{fig:configuration}
\vspace{-3mm}
\end{figure}

Figure~\ref{fig:configuration} depicts a snippet of \conesc code to
implement the ``Base Station'' group in Figure~\ref{fig:design}. In
this example, the \code{report()} command on
line~\lstref{cg:layered}---used to report a contact with another animal
to the end-user---changes the behavior depending on whether the
base-station is \code{Reachable} or \code{Unreachable}. The latter are
the contexts included in this group, specified on line~\lstref{cg:ctx}
after the keyword \code{contexts} with optional modifiers: \code{is
  default} (line~\lstref{cg:def}) specifies the active context at
start-up, and \code{is error} (line~\lstref{cg:error}) indicates an
error context. The latter is automatically activated should there be
violations to constraints defined over context transitions, e.g., the
fact that ``Resting'' or ``NotMoving'' must be active when
transitioning from ``Healthy'' to ``Diseased'', as shown in
Figure~\ref{fig:design}.

\begin{figure}[!tb]
\begin{lstlisting}[style=conescframe]
context Reachable {
*\lstnote{ct:tr}* transitions Unreachable;
*\lstnote{ct:trigger}* triggers NotMoving;
 uses interface Radio;
} implementation {
*\lstnote{ct:layer}* layered command void report(contact_t contact){
  call Radio.send(contact);}}
*\lstnote{ct:activate}* event void activated(){// Dump logs on base-station }
*\lstnote{ct:deactivate}* event void deactivated(){ // Radio clean-up }
\end{lstlisting}
\vspace{-5mm}
\caption{Individual context in \conesc.}
  \label{fig:context}
\vspace{-5mm}
\end{figure}

Figure~\ref{fig:context} shows the \conesc specification of the
``Reachable'' context. The keyword \code{transitions} on
line~\lstref{ct:tr} specifies the allowed outgoing transitions, whereas
the keyword \code{triggers} on line~\lstref{ct:trigger} is available
to bind context activations across groups. In this case, entering the
``Unreachable'' context consequently activates ``NotMoving'', as
specified in Figure~\ref{fig:design}. The specific implementation of
the layered function is indicated on line~\lstref{ct:layer} with the
\code{layered} keyword.  The implementation of other commands or
events is as in standard nesC. Particularly, the predefined events
\code{activated()} and \code{deactivated()}, shown on
lines~\lstref{ct:activate} and~\lstref{ct:deactivate}, are
automatically signalled when entering and existing the context,
allowing the implementation of one-time operations and the
starting/stopping of continuous activities in a context.

The contexts indicated after the \code{transitions} keyword inside
individual contexts are possibly followed by the \code{iff} keyword to
state constraints on the transitions, as in
\begin{lstlisting}[language=conesc]
transitions Diseased iff Resting || NotMoving;
\end{lstlisting}
used in the definition of the ``Healthy'' context to encode the
constraints in Figure~\ref{fig:design}. If such a transition is
attempted at run-time, but the constraints are violated, the error
context defined in the corresponding context group is activated. 

Explicit context activation may happen anywhere in the code, using the
\code{activate} keyword, as in
\begin{lstlisting}[language=conesc]
activate BaseStationG.Unreachable;
\end{lstlisting}
This enables complete decoupling between the context-de\-pen\-dent
application logic---confined within the layered functions inside the
individual contexts---and the adaptation logic itself, which may be
specified in a separate component. Such feature helps separate
orthogonal concerns and hence facilitates testing, maintenance,
and evolution of the software, as we discuss next.

%%% Local Variables: 
%%% mode: latex
%%% TeX-master: "paper"
%%% End: 

\section{Current Work and Preliminary Evaluation}
\label{sec:eval}
\lm{High choesion, but components not enough...}

We implemented a translator from \conesc to pure nesC that allows us
to rely on the optimized nesC tool-chain to obtain running
implementations. As for tool support, we are also investigating ways
to automatically generate \conesc skeletons based on graphical
representations of contexts and context groups similar to
Figure~\ref{fig:design}. We are also planning to use the same notation
as input to perform domain-specific static verification, e.g., using
model-checking techniques~\cite{mottolaicse}.

Using the translator, we compare an implementation of the wildlife
monitoring application using our approach against a
functionally-equivalent implementation that would arguably result from
current practice~\cite{programmingsurvey,badgersEWSN}. First, we study
the logical structuring and simplicity of the implementations, which
determine to a great extent the ease of debugging and
maintenance. Next, we study to what extent either approach lends
itself to evolving the software against initially unforeseen
requirements. Finally, we assess the system overhead that our approach
introduces over the current practice, which represents the price to
pay to obtain the advantages it brings.

% \hrule
% \noindent Three key aspects:
% \begin{itemize}
% \item qualitatively compare ConesC and nesC implementations along
%   coupling, dependencies, programmer-visible program states, ...
% \item demonstrates that changing the application (for example, adding
%   a new context) is easier in ConesC than nesC;
% \item shows a few figures on run-time overhead to demonstrate the
%   approach is feasible.
% \end{itemize}

\hrule
The results we present hereafter are based on a TMote
Sky~\cite{} platform, a wireless sensor node featuring a TI MSP430 MCU
with 8KBytes of RAM and a low-power CC2420 radio.
\hrule

%%% Local Variables: 
%%% mode: latex
%%% TeX-master: "paper"
%%% End: 

\section{Related Work}
\label{sec:related}

The very first work towards context-oriented programming was provided by
Hirschfeld et al.~\cite{Hirschfeld08}, while the main language constructs of COP
were provided by Costanza et al.~\cite{Costanza05}. Some work is related to
developing of context-oriented languages~\cite{Sehic11,Kamina11,Bardram05},
but they are based, however, on high-level languages and resource consuming
platforms, and not applicable for autonomous CPSs.

There are some attempts to apply context-oriented programming for the software
for CPSs. Sehic et al.~\cite{Sehic11}, for example, proposed a macro-language
for rapid development of context-aware applications for CPSs. Wood et
al.~\cite{Wood08} developed a CPS for Assisted-Living and Residential
Monitoring, which utilizes a technique of a context-depended behavior. Another
work, provided by Ghezzi et al.~\cite{Ghezzi10}, is devoted to the language
support for bringing a context-awareness for CPSs.

Several attempts were also performed towards self-adaptive software for CPSs. In
the work of Steine et al.~\cite{Steine11} authors were focused on run-time
reconfiguration of WSNs by adjusting protocol values dynamically depending on
the conditions the system operates in. Marques et al.~\cite{Marques11} propose a
run-time evaluation of QoS characteristics for better adaptation. Another
research was provided towards self-adaptive protocols: MAC-protocol by Park et
al.~\cite{Park08} and routing protocol by Bourdenas et al.~\cite{Bourdenas11}.
Self-adaptivity is also considered as a main ingredient of run-time error
handling~\cite{Bourdenas10} and energy management~\cite{Jiang07}.

Despite the research activity in the area of self-adaptive software for CPSs is
quite intensive, it remains fragmented, and no holistic approach yet developed.
Current research has two extreme directions:~\emph{i)} high-level software,
which brings adaptivity only in application level using CPSs as raw-data
providers~\cite{Wood08,Sehic11,Ghezzi10}, and~\emph{ii)} low-level software,
which is focused on a system level ignoring an application
level~\cite{Steine11,Marques11,Park08,Bourdenas11,Bourdenas10,Jiang07}.
We combine different aspects of self-adaptivity for CPSs and provide robust
approach for self-adaptivity on both system and application levels, which makes
our work novel in this field.

Salvaneschi et al.~\cite{SalvaneschiTBP}\footnote{to be published}\cite{Salvaneschi12} 
outlined possible techniques for language adaptation for context-oriented
programming. We not only utilize concepts of COP, but also provide a novel approach for
bringing context-oriented programming to CPSs. To this end, we modified and
applied such techniques as: \emph{Per-Object activation} and
\emph{Layer-in-Class} \cite{Salvaneschi12}. The latter implies the
implementations of layers~\cite{Costanza05} -- i.e. behavioral variations -- in
one class. In our approach we used \emph{context groups} to aggregate behavioral
variations, which are invoked by explicit activation of corresponding
\emph{context}, i.e. we adopted \emph{Per-Object activation}, where an
\emph{object} is a~\emph{context}.

\section{Conclusion}
\label{sec:ending}

We presented design-time and programming support for self-adaptive
software in resource-constrained CPSs. In this domain, the lack of a
principled design approach and the existing rudimentary programming
environments result in entangled implementations. To address this
issue, we conceived dedicated design concepts and COP extensions to
low-level CPS programming languages. Preliminary results indicate that
our approach yields better structured implementations that are easier
to test, maintain, and evolve. The run-time overhead to pay is to gain
these advantages is, nonetheless, negligible.

%%% Local Variables: 
%%% mode: latex
%%% TeX-master: "paper"
%%% End: 


% {\small
\bibliographystyle{plain}      % 
\bibliography{bibl}   % name your BibTeX data base
% }

\end{document}


