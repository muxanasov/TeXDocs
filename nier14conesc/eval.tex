\section{Preliminary Evaluation}
\label{sec:eval}

Using the translator, we compare an implementation of the wildlife
monitoring application against a functionally-equivalent nesC
implementation that would arguably result from current
practice~\cite{mottola10:survey,Picco:2010:SEW:1882362.1882421,pasztor10:selective}. Nonetheless,
the following considerations more generally apply to a larger set of
applications we are experimenting with, including a smart-home
controller and an adaptive sensor network protocol stack, whose
detailed discussion we omit for brevity. Our comparison addresses
three key dimensions, as described next.

% First, we
% study the logical structuring and simplicity of the implementations,
% which determine to a great extent the ease of debugging and
% maintenance. Next, we study to what extent either approach lends
% itself to evolving the software against initially unforeseen
% requirements. Finally, we assess the system overhead our approach
% introduces over the current practice.

\begin{figure}[!tb]
\renewcommand{\arraystretch}{1.1}
\scriptsize
\centering
\centering
\begin{tabular}{|l|p{2.5in}|}
\hline
\bfseries Type & \bfseries Description\\
\hline
Content (tightest) & One module relies on the internal working of another. Chang- ing one module requires changes in the other as well.\\
\hline
Common & Two or more modules share some global state, e.g., a variable.\\
\hline
External & Two or more modules share a common data format.\\
\hline
Control & One module controls the flow of another, e.g., passing infor- mation that determine how to execute.\\
\hline
Stamp & Two or more modules share a common data format, but each of them uses a different part with no overlapping.\\
\hline
Data & Two or more modules share data through a typed interface, e.g., a function call.\\
\hline
Message (loosest) & Two or more modules share data through an untyped inter- face, e.g., via message passing.\\
\hline
\end{tabular}
\vspace{-2mm}
\caption{Coupling types.}
\label{tab:couptypes}
\vspace{-7mm}
\end{figure}

% \hrule
% \noindent Three key aspects:
% \begin{itemize}
% \item qualitatively compare ConesC and nesC implementations along
%   coupling, dependencies, programmer-visible program states, ...
\fakepar{1.\ Coupling and cohesion} According to Stevens et
al.~\cite{Stevens79}, seven types of coupling between software components
exist, as summarized in Figure~\ref{tab:couptypes}. It is generally
known that the tightest is coupling, the more difficult is debugging,
maintaining, and extending the implementations.

Our analysis reveals that the \conesc implementation is significantly
more decoupled than its nesC counterpart, as qualitatively indicated
in Figure~\ref{fig:coupres}. Based on current practice, the adaptation
functionality would be implemented as a monolithic nesC component,
using global state variables to dispatch function calls to different
components depending on the situation. With our approach, most
types of coupling among components are removed, as context transitions
implicitly cater for the appropriate dispatching of function
calls. Consequently, programmers are relieved from explicitly managing
global state transitions, which facilitates maintenance and testing.


\begin{figure}[!tb]
\renewcommand{\arraystretch}{1.1}
\scriptsize
\centering
\begin{tabular}{|l|l|l|l|l|l|l|l|}
\hline
\bfseries Implementation & \rotatebox{90}{\bfseries Content} & \rotatebox{90}{\bfseries Common} 
& \rotatebox{90}{\bfseries External} & \rotatebox{90}{\bfseries Control}
& \rotatebox{90}{\bfseries Stamp} & \rotatebox{90}{\bfseries Data}
& \rotatebox{90}{\bfseries Message}\\
\hline
nesC-based &
yes&yes&yes&yes&--&yes&--\\
\hline
ConesC-based &
--&--&yes&--&--&yes&--\\
\hline
\end{tabular}
\vspace{-2mm}
\caption{Coupling comparison.}
\vspace{-7mm}
\label{fig:coupres}
\end{figure}

The individual \conesc components also appear more
cohesive and simpler: on average, using our approach, a component is
almost half the lines of code than using plain nesC and defines half
the number of global variables compared to nesC. We further observe a 75\%
reduction in the number of commands/events declared in \conesc
components compared to a plain nesC implementation.

% Our analysis shows that ConesC implementation is much more decoupled as
% compared to its nesC counterpart. Since each context represents a situation
% the system may find itself in, they are isolated into modules and do not rely
% on the internal work of each other or share any global state, which reduces
% programmer-visible states. Contrary, in nesC implementation, each situation
% is represented by a global state, which should have been taken into account
% in every module.

% \item demonstrates that changing the application (for example, adding
%   a new context) is easier in ConesC than nesC;

\fakepar{2.\ Evolving the software} Besides dealing with the run-time
adaptation required to cope with environmental dynamics, CPS software
also needs to evolve in response to changing
requirements~\cite{Picco:2010:SEW:1882362.1882421}. Generally, the
better an implementation is modularized, the easier are the
modifications, since the changes are limited to isolated portions of
the system.

Say, for example, developers of the wildlife monitoring applications
need to track the spreading of a disease. To this end, they modify the
application design by adding a new context ``Carrier'' in the ``Health
conditions'' group to mark an animal who was in contact with a
diseased one, but shows no increase in body temperature yet.

Using our approach, this is as simple as changing 5~lines of code in
the \conesc implementation, besides the functionality needed in a
new ``Carrier'' context.  Based on current practice, the same
modification would crucially require adding at least two global
states.  This adds the burden of explicitly managing their transitions
and is further detrimental to ease of testing and maintainability.

\fakepar{3.\ System overhead} The advantages above come at the price
of a negligible system overhead. We assess this aspect on the
widespread TMote Sky sensor node~\cite{polastre05telos}, featuring a
16-bit MSP430 MCU with 10 Kb RAM and 48Kb for program memory. In
particular, we measure the MCU overhead for context transitions and
calls to layered functions, as well as memory overhead when using
\conesc as compared to nesC. To measure the MCU overhead we use the
MSPSim MSP430 emulator [5], while we estimate the memory overhead
using the tools in the nesC and GNU-C toolchains.

Aboard the TMote Sky, calls to layered functions in our example
application add only 3 CPU cycles over standard function calls,
whereas context transitions involve at most 20 CPU cycles. To put
these figures in prospective, turning an LED on takes 8 CPU cycles on
a TMote Sky. The increased latency and consumed energy due to the
additional CPU cycles are hence immaterial. As for memory consumption,
we measure a mere 3\% increase in both RAM and program memory. These
figures demonstrate that the price to pay for obtaining significant
advantages in the quality of the implementations is modest, and hence
our approach promises to be feasible even on extremely
resource-constrained platforms.

% \item shows a few figures on run-time overhead to demonstrate the
%   approach is feasible.
% % \end{itemize}

% ConeC approach in our scenario shows a CPU-overhead up to 3 cycles for calls of
% a layered function and up to 20 CPU-cycles for context transitions, depending on
% the number of contexts in a context group, where the function and contexts are
% declared. These numbers are negligible in terms of energy consumption, since the
% simplest operation, like enabling/disabling LED, uses 8 CPU-cycles.



%%% Local Variables: 
%%% mode: latex
%%% TeX-master: "paper"
%%% End: 
