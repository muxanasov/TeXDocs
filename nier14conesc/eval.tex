\section{Current Work and Preliminary Evaluation}
\label{sec:eval}

We implemented a translator from \conesc to pure nesC that allows us
to rely on the optimized nesC tool-chain to obtain running
implementations. As for tool support, we are also investigating ways
to automatically generate \conesc skeletons based on graphical
representations of contexts and context groups similar to
Figure~\ref{fig:design}. We are also planning to use the same notation
as input to perform domain-specific static verification, e.g., using
model-checking techniques~\cite{mottolaicse}.

Using the translator, we compare an implementation of the wildlife
monitoring application using our approach against a
functionally-equivalent implementation that would arguably result from
current practice~\cite{programmingsurvey,badgersEWSN}. First, we study
the logical structuring and simplicity of the implementations, which
determine to a great extent the ease of debugging and
maintenance. Next, we study to what extent either approach lends
itself to evolving the software against initially unforeseen
requirements. Finally, we assess the system overhead that our approach
introduces over the current practice, which represents the price to
pay to obtain the advantages it brings.

% \hrule
% \noindent Three key aspects:
% \begin{itemize}
% \item qualitatively compare ConesC and nesC implementations along
%   coupling, dependencies, programmer-visible program states, ...
% \item demonstrates that changing the application (for example, adding
%   a new context) is easier in ConesC than nesC;
% \item shows a few figures on run-time overhead to demonstrate the
%   approach is feasible.
% \end{itemize}

\hrule
The results we present hereafter are based on a TMote
Sky~\cite{} platform, a wireless sensor node featuring a TI MSP430 MCU
with 8KBytes of RAM and a low-power CC2420 radio.
\hrule

%%% Local Variables: 
%%% mode: latex
%%% TeX-master: "paper"
%%% End: 
