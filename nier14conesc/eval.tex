\section{Current Work and Preliminary Evaluation}
\label{sec:eval}

We implemented a translator from \conesc to pure nesC that allows us
to rely on the nesC tool-chain to obtain running
implementations. We are also investigating ways
to automatically generate \conesc skeletons based on graphical
representations of contexts and context groups similar to
Figure~\ref{fig:design}. Further, we plan to use the same notation
as input to perform static verification, e.g., using domain-specific
model-checking techniques~\cite{mottolaicse}.

Using the translator, we compare an implementation of the wildlife
monitoring application against a functionally-equivalent nesC
implementation that would arguably result from current
practice~\cite{programmingsurvey,badgersEWSN}. Our comparison
addresses three key dimensions, as described next.

% First, we
% study the logical structuring and simplicity of the implementations,
% which determine to a great extent the ease of debugging and
% maintenance. Next, we study to what extent either approach lends
% itself to evolving the software against initially unforeseen
% requirements. Finally, we assess the system overhead our approach
% introduces over the current practice.

\begin{figure}[!tb]
\renewcommand{\arraystretch}{1.1}
\scriptsize
\centering
\begin{tabular}{|l|l|l|l|l|l|l|l|}
\hline
\bfseries Implementation & \rotatebox{90}{\bfseries Content} & \rotatebox{90}{\bfseries Common} 
& \rotatebox{90}{\bfseries External} & \rotatebox{90}{\bfseries Control}
& \rotatebox{90}{\bfseries Stamp} & \rotatebox{90}{\bfseries Data}
& \rotatebox{90}{\bfseries Message}\\
\hline
nesC-based &
yes&yes&yes&yes&--&yes&--\\
\hline
ConesC-based &
--&--&yes&--&--&yes&--\\
\hline
\end{tabular}
\caption{Coupling comparison.}
\vspace{-5mm}
\label{fig:coupres}
\end{figure}

% \hrule
% \noindent Three key aspects:
% \begin{itemize}
% \item qualitatively compare ConesC and nesC implementations along
%   coupling, dependencies, programmer-visible program states, ...
\fakepar{1.\ Coupling and cohesion} Our analysis reveals that the
\conesc implementation is significanlty more decoupled than its nesC
counterpart, as shown in Figure~\ref{fig:coupres}. Based on current
practice, the adaptation functionality would be implemented as a
monolithic nesC component, using global state variables to dispatch
function calls depending on the current situation. With our approach,
most types of coupling among components are removed, as context
transitions implicitly cater for the needed dispatching of function
calls. Consequently, programmers are relieved from explicitly managing
global state transitions, which facilitates maintenance and testing.

The individual \conesc components also appear more
cohesive and simpler: on average, using our approach, a component is
almost half the lines of code than using plain nesC and defines half
the number of global variables as in nesC. We further observe a 75\%
reduction in the number of commands/events declared in \conesc
components compared to a plain nesC implementation.

% Our analysis shows that ConesC implementation is much more decoupled as
% compared to its nesC counterpart. Since each context represents a situation
% the system may find itself in, they are isolated into modules and do not rely
% on the internal work of each other or share any global state, which reduces
% programmer-visible states. Contrary, in nesC implementation, each situation
% is represented by a global state, which should have been taken into account
% in every module.

% \item demonstrates that changing the application (for example, adding
%   a new context) is easier in ConesC than nesC;

\fakepar{2.\ Evolving the software} CPS software needs to
evolve in response to unforeseen environment dynamics and changing
requirements. Say, for example, developers of the wildlife monitoring
applications need to track the spreading of a disease. To this
end, they modify the application design by adding a new context
``Carrier'' to mark an animal who was in contact with a diseased one,
but shows no increase in body temperature yet. 

Using our approach, this is as simple as changing 5~lines of code in
the \conesc implementation, besides the functionality needed in the
new \code{Carrier} context.  Based on current practice, the same
modification would crucially require adding at least two global
states.  This adds the burden of explicitly managing their
transitions and is further detrimental to maintainability and ease
of debugging.

\fakepar{3.\ System overhead} The advantages above come at the price
of a negligible system overhead. On the widespread TMote Sky sensor
node~\cite{}, calls to layered functions in our example application
add only 3 CPU cycles over standard function calls, whereas context
transitions involve at most 20 CPU cycles. To put these figures in
prospective, turning an LED on takes 8 CPU cycles on a TMote Sky
node. The energy consumed by the additional CPU cycles is hence
immaterial. As for memory consumption, we measure a mere 3\% increase
in both RAM and program memory. These figures demonstrate that the
price to pay for obtaining signicant advanatges in the quality of the
resulting implementations is very modest, and hence our approach is
ultiamtely feasible even on extremely resource-cosntrained platforms.

% \item shows a few figures on run-time overhead to demonstrate the
%   approach is feasible.
% % \end{itemize}

% ConeC approach in our scenario shows a CPU-overhead up to 3 cycles for calls of
% a layered function and up to 20 CPU-cycles for context transitions, depending on
% the number of contexts in a context group, where the function and contexts are
% declared. These numbers are negligible in terms of energy consumption, since the
% simplest operation, like enabling/disabling LED, uses 8 CPU-cycles.



%%% Local Variables: 
%%% mode: latex
%%% TeX-master: "paper"
%%% End: 
