\newsavebox{\boxcc}
\Savebox{\boxcc}{
\hspace{-0.25cm}
\begin{minipage}[l]{\columnwidth}
\begin{lstlisting}[style=conescframe]
context group BaseStationG {
*\lstnote{layereddef}* layered command void report(msg_t msg);
}implementation {
*\lstnote{contexts}* contexts Reachable,
*\lstnote{isdefault}*          Unreachable is default,
*\lstnote{iserror}*          ErrorC is error;
 components Routing, Logging, LedsC;
 Reachable.Collection -> Routing;
 Unreachable.DataStore -> Logging;
 ErrorC.Leds -> LedsC;}
\end{lstlisting}
\end{minipage}
}
\newsavebox{\boxbscm}
\Savebox{\boxbscm}{
\hspace{-0.25cm}
\begin{minipage}[l]{\columnwidth}
\begin{lstlisting}[style=conescframe]
module BaseStationContextManager {
*\lstnote{cgdecl}* uses context group BaseStationG;
}implementation {
 event msg_t Radio.receive(msg_t msg) {
*\lstnote{actBS}*  activate BaseStationG.Reachable;
  call BSReset.startOneShot(INTERVAL);}
 event void BSReset.fired() {
*\lstnote{actNoBS}*  activate BaseStationG.Unreachable;}}
\end{lstlisting}
\end{minipage}
}
\newsavebox{\boxc}
\Savebox{\boxc}{
\hspace{-0.25cm}
\begin{minipage}[l]{\columnwidth}
\begin{lstlisting}[style=conescframe]
context Unreachable {
*\lstnote{dependence}* transitions InRange iff ActivityG.Running;
 uses interface DataStore;
}implementation {
*\lstnote{activated}* event void activated(){//...}
*\lstnote{deactivated}* event void deactivated(){//...}
*\lstnote{layeredimp}* layered command void report(msg_t msg){
  call DataStore.deposit(msg);}}
\end{lstlisting}
\end{minipage}
}
\newsavebox{\boxirc}
\Savebox{\boxirc}{
\hspace{-0.25cm}
\begin{minipage}[l]{\columnwidth}
\begin{lstlisting}[style=conescframe]
context Reachable {
 uses interface Collection;
 uses context group BatteryG; 
}implementation {
*\lstnote{check}* command bool check(){
  return call BatteryG.getContext() == BatteryG.Normal;}
*\lstnote{layeredimp2}* layered command void report(msg_t msg){
  call Collection.send(msg);}}
\end{lstlisting}
\end{minipage}
}
\newsavebox{\boxlc}
\Savebox{\boxlc}{
\hspace{-0.25cm}
\begin{minipage}[l]{\columnwidth}
\begin{lstlisting}[style=conescframe]
context Low {
*\lstnote{triggers}* triggers BaseStationG.OutRange;
}implementation {//...}
\end{lstlisting}
\end{minipage}
}
\newsavebox{\boxmc}
\Savebox{\boxmc}{
\hspace{-0.25cm}
\begin{minipage}[l]{\columnwidth}
\begin{lstlisting}[style=conescframe]
configuration ApplicationC {
}implementation {
*\lstnote{declaration}* components BaseStationG, Application;
 //...
*\lstnote{wiring}* Application.BaseStationG -> BaseStationG;}
\end{lstlisting}
\end{minipage}
}
\newsavebox{\boxmm}
\Savebox{\boxmm}{
\hspace{-0.25cm}
\begin{minipage}[l]{\columnwidth}
\begin{lstlisting}[style=conescframe]
module Application {
*\lstnote{cgdecl}* uses context group BaseStationG;
}implementation {
 event void Timer.fired() {
*\lstnote{calling}*  call BaseStationG.report(msg);}
*\lstnote{eventCC}* event void BaseStationG.contextChanged(context_t con) {
*\lstnote{concheck}*  if(con == BaseStationG.Reachable)
   dbg("Debug", "Base station is reachable!");}}
\end{lstlisting}
\end{minipage}
}
\newsavebox{\boxnmc}
\Savebox{\boxnmc}{
\hspace{-0.25cm}
\begin{minipage}[l]{\columnwidth}
\begin{lstlisting}[style=conescframe]
context NotMoving {
*\lstnote{transitions}* transitions Resting;
}implementation {//...}
\end{lstlisting}
\end{minipage}
}
\section{Programming Support}
\label{sec:conesc}

We render the design concepts above in a set of context-oriented
programming (COP)~\cite{Hirschfeld08} constructs feasible within
existing programming environments for resource-constrained CPS
platforms~\cite{mottola10:survey}.  We exemplify our approach based on
nesC~\cite{gay03nesc}, a mainstream sensor network programming
language. However, our approach is not tied to it, and may be readily
translated to other similar programming systems~\cite{mottola10:survey}.

\fakepar{Target language} nesC is a component-based event-driven
programming framework for sensor networks, derived from
C. Applications are built by interconnecting \emph{components} that
interact by providing or using \emph{interfaces}. An interface lists
one or more functions, tagged as \emph{commands} or
\emph{events}. Commands are used to execute actions, while events are
used to collect the results asynchronously.  Interfaces in nesC are
bidirectional: data flows both ways between components connected
through the same interface.  Component \emph{configurations} specify
the wirings among components. Configurations are component themselves,
so they can provide interfaces and be wired to other components.

nesC exemplifies the limitations dictated by the target platforms, and
hence the reasons why existing COP approaches cannot be directly
ported. Besides the inability to create run-time instances of
components and to reconfigure component wirings we already mentioned,
for example, the use of dynamically-allocated memory is also
discouraged: the MCUs provide no memory protection, so bugs in memory
handling may have disastrous effects.

\fakepar{\conesc} Notwithstanding the above constraints, we design a
context-oriented extension to nesC, called \conesc, that incorporates
the design concepts described in Section~\ref{sec:design}. 

At the core of \conesc is a notion of \emph{layered
  function}~\cite{Hirschfeld08}. These are functions whose behavior
depends on the currently active context, and are hence the primary
means to implement the behavioral variations necessary for
self-adaptation. Crucially, the behavior of layered functions may
change \emph{transparently} to the caller, which is then relieved from
explicitly managing the adaptation required by the situation.


\begin{figure}[!tb]
\begin{lstlisting}[style=conescframe]
context group BaseStationG {
*\lstnote{cg:layered}* layered command void report(contact_t contact);
}
implementation {
*\lstnote{cg:ctx}* contexts Reachable, 
*\lstnote{cg:def}*          Unreachable is default,
*\lstnote{cg:error}*          ErrorC is error;
 // Standard nesC component wirings... 
}
\end{lstlisting}
\vspace{-4mm}
\caption{Context group in \conesc.}
  \label{fig:configuration}
\vspace{-5mm}
\end{figure}

A \emph{context group} in \conesc extends nesC configurations
by specifying the contexts included in the group and the layered
functions that such contexts provide.  Figure~\ref{fig:configuration}
shows a snippet of \conesc code to implement the ``Base Station''
group in Figure~\ref{fig:design}. In this example, the \code{report()}
command on line~\lstref{cg:layered}---used to report a contact with
another animal to the end-user---changes the behavior depending on
whether the base-station is \code{Reachable} or
\code{Unreachable}. The latter are the contexts included in this
group, specified on line~\lstref{cg:ctx} after the keyword
\code{contexts} with optional modifiers: \code{is default}
(line~\lstref{cg:def}) specifies the active context at start-up, and
\code{is error} (line~\lstref{cg:error}) indicates an error
context. The latter is automatically activated should there be
violations to constraints defined over context transitions; for
example, the fact that ``Resting'' or ``NotMoving'' must be active
when transitioning from ``Healthy'' to ``Diseased'', as shown in
Figure~\ref{fig:design}.

\begin{figure}[!tb]
\begin{lstlisting}[style=conescframe]
context Reachable {
*\lstnote{ct:tr}* transitions Unreachable;
*\lstnote{ct:trigger}* triggers NotMoving;
 uses interface Radio;
} 
implementation {
*\lstnote{ct:layer}* layered command void report(contact_t contact){
  call Radio.send(contact);
}
*\lstnote{ct:activate}* event void activated(){// Dump logs on base-station }
*\lstnote{ct:deactivate}* event void deactivated(){ // Radio clean-up }
}
\end{lstlisting}
\vspace{-4mm}
\caption{\conesc context.}
  \label{fig:context}
\vspace{-7mm}
\end{figure}

Contexts extend nesC components by providing the context-dependent
implementations of layered functions.  Figure~\ref{fig:context} shows
the \conesc specification of the ``Reachable'' context. The keyword
\code{transitions} on line~\lstref{ct:tr} specifies the allowed
outgoing transitions, whereas the keyword \code{triggers} on
line~\lstref{ct:trigger} binds context activations across groups. In
this case, entering the ``Unreachable'' context consequently activates
``NotMoving'', as specified in Figure~\ref{fig:design}. The specific
implementation of the layered function is indicated on
line~\lstref{ct:layer} with the \code{layered} keyword.  The
implementation of other commands or events is as in standard
nesC. Particularly, the predefined events \code{activated()} and
\code{deactivated()}, shown on lines~\lstref{ct:activate}
and~\lstref{ct:deactivate}, are automatically signalled when entering
and existing the context, allowing the implementation of one-time
operations and the starting/stopping of continuous activities in a
context.

The contexts indicated after the \code{transitions} keyword inside
single contexts are possibly followed by the \code{iff} keyword to
state constraints on the transitions, as in
\begin{lstlisting}[language=conesc]
transitions Diseased iff Resting || NotMoving;
\end{lstlisting}
used in the definition of the ``Healthy'' context to encode the
constraints in Figure~\ref{fig:design}. If such a transition is
attempted at run-time, but the constraints are violated, the error
context defined in the corresponding context group is activated. 


\putsnippet{
 caption=Base-station context controller.,
 label=fig:bscm,
 boxname=boxbscm
}

Programmers can, anywhere in the code, trigger explicit transitions
between contexts in a group. This is as simple as using the
\code{activate} keyword followed by a full context name, as shown in
Figure~\ref{fig:bscm}. The full context name is determined by
concatenating the name of the enclosing context group and of the
single context. For example, the \emph{Reachable} context is activated
on line~\lstref{actBS} of Figure~\ref{fig:bscm} as soon as a beacon
from the base station is received. Should the timeout expire with no
more beacons received, context \emph{Unreachable} is activated on
line~\lstref{actNoBS}. Either context change results in a different
context-dependent implementation of \code{report} to be
activated. These implementations are found within the single contexts.

\putsnippet{
 caption=Caller module.,
 label=fig:mm,
 boxname=boxmm
}

Modules using layered functions perform function calls transparently
w.r.t.\ the available contexts and, most importantly, independently of
what context is active at a given moment. Fig.~\ref{fig:mm} shows one
such example for function \code{report}. Following the indication that
context group \emph{BaseStationG} is used, as specified on
line~\lstref{cgdecl}, the call to the layered function \code{report}
does not refer to the single contexts. The net advantage is that
the use of context-dependent functionality is fully decoupled w.r.t.\
context detection and activation. The two may be implemented even in
different modules. Such feature helps separate
orthogonal concerns and hence facilitates testing, maintenance,
and evolution of the software, as we discuss next.

Should programmers of caller modules need to find out
about context changes, a predefined event~\code{contextChanged} is
fired corresponding to every context change, as in
line~\lstref{eventCC} in Fig.~\ref{fig:mm}. % This event can be caught
% and handled, as it is shown on the line~\lstref{eventCC} in
% Fig.~\ref{fig:mm}, but it is not mandatory though.
Within the event handler, programmers can access constant values that
our translator automatically generates to find out what context was
activated and to react accordingly, as shown on
line~\lstref{concheck}.


% Explicit context activation may happen anywhere in the code, using the
% \code{activate} keyword, as in
% \begin{lstlisting}[language=conesc]
% activate BaseStationG.Unreachable;
% \end{lstlisting}
% This enables complete decoupling between the context-de\-pen\-dent
% application logic---confined within the layered functions inside the
%  contexts---and the adaptation logic itself, which may be
% specified in a separate component. 

\fakepar{Translation} We develop a dedicated translator to convert
\conesc code to plain nesC. To do so, our translator performs two
passes through the input code. First, it reads the main nesC
\code{Makefile} to determine the main configuration component and to
recursively scan the component tree. Based on the information gained
during the first pass, including the list of every context and context
groups defined in the code, it parses every input file to convert the
\conesc code to plain nesC and to generate a set of support
functionality managing context transitions at run-time. The resulting
sources are then compiled using the standard nesC toolchain.

Our translator is implemented using JavaCC~\cite{javacc}. Two aspects
are worth noticing. First, the generated code is completely
hardware-independent. Therefore, hardware compatibility is the same as
the original nesC toolchain, allowing us to support a wide range of
platforms and not to modify our translator due to hardware
idiosyncrasies. Second, the whole translation process is only
seemingly straightforward. Rendering the logic embedded within the
\conesc abstractions does require a fairly sophisticated
processing. To give an intuition, we measured the size of the \conesc
implementations of a set of representative applications against the
size of the nesC implementations output by our translator. On average,
we observe three times as much lines of code in the
automatically-generated nesC code.

%%% Local Variables: 
%%% mode: latex
%%% TeX-master: "paper"
%%% End: 
