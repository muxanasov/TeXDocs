\section{Conclusion and Future Work}
\label{sec:ending}

We presented a solution to provide design-time and programming support
for self-adaptive software in re\-sour\-ce-constrained component-based
CPS software. In this domain, the lack of a principled design approach
and the rudimentary programming environments result in entangled
implementations. To remedy this, we conceived dedicated design
concepts and COP extensions to component-based frameworks for
resource-constrained CPSs. Preliminary results indicate that our
approach yields implementations that are easier to test, maintain, and
evolve. The run-time overhead to pay is, nonetheless, negligible.

% Our immediate research agenda includes automatic generation of \conesc
% code templates from graphical notations akin to
% Figure~\ref{fig:design}, and domain-specific model checking of
% context-oriented CPS applications built based on our approach.

% \hrule

As part of our research agenda, we are investigating ways to
automatically generate \conesc skeletons based on graphical
representations of contexts and context groups similar to
Figure~\ref{fig:design}. Further, we plan to use the same notation as
input to perform static verification, e.g., using domain-specific
model-checking techniques.

%%% Local Variables: 
%%% mode: latex
%%% TeX-master: "paper"
%%% End: 
