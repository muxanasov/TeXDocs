\section{Related Work}
\label{sec:related}

The very first work towards context-oriented programming was provided by
Hirschfeld et al.~\cite{Hirschfeld08}, while the main language constructs of COP
were provided by Costanza et al.~\cite{Costanza11}. There are a lot of work
related to developing of context-oriented
languages~\cite{Sehic11,Kamina11,Bardram05}, but they are based, however, on
high-level languages and resource consuming platforms, and not applicable for
autonomous CPSs.

There are some attempts to apply context-oriented programming for the software
for CPSs. Sehic et al.~\cite{Sehic11} proposed, for example, a macro-language
for rapid development of context-aware applications for CPSs. Wood et
al.~\cite{Wood08} developed a CPS for Assisted-Living and Residential
Monitoring, which utilizes a technique of a context-depended behavior. Another
work, provided by Ghezzi et al.~\cite{Ghezzi10}, is devoted to the language
support for bringing a context-awareness for CPSs.

Several attempts were also performed towards self-adaptive software for CPSs. In
the work of Steine et al.~\cite{Steine11} authors were focused on run-time
reconfiguration of WSNs by adjusting protocol values dynamically depending on
the conditions the system operates in. Marques et al.~\cite{Marques11} propose a
run-time evaluation of QoS characteristics for better adaptation. Another
research was provided towards self-adaptive protocols: MAC-protocol by Park et
al.~\cite{Park08} and routing protocol by Bourdenas et al.~\cite{Bourdenas11}.
Self-adaptivity is also considered as a main ingredient of run-time error
handling~\cite{Bourdenas10} and energy management~\cite{Jiang07}.

Despite the research activity in the area of self-adaptive software for CPSs is
quite intensive, it remains fragmented, and no holistic approach yet developed.
Researches are usually focused only on the
system~\cite{Steine11,Marques11,Park08,Bourdenas11,Bourdenas10,Jiang07} or only
on the application level~\cite{Wood08,Sehic11,Ghezzi10}, while the another level is
absolutely abandoned. In our work we combine different aspects of
self-adaptivity for CPSs and provide robust approach for self-adaptivity on both
system and application levels.

Salvaneschi et al.~\cite{Salvaneschi11,Salvaneschi12} outlined possible
techniques for language adaptation for context-oriented programming. Our work
not only utilizes concepts of COP, but also provides a novel approach for
bringing context-oriented programming to CPSs. To this end, we modified and
applied such techniques as: \emph{Per-Object activation} and
\emph{Layer-in-Class} \cite{Salvaneschi12}. The latter implies the
implementations of layers~\cite{Costanza11} -- i.e. behavioral variations -- in
one class. In our approach we used \emph{context groups} to aggregate behavioral
variations, which are invoked by explicit activation of corresponding
\emph{context}, i.e. we adopted \emph{Per-Object activation}, where an
\emph{object} is a~\emph{context}.
