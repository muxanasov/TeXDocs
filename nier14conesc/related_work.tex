\section{Related Work}
\label{sec:related}

Within the field of self-adaptive software~\cite{cheng:adaptive},
efforts close to ours particularly address the design and
implementation of self-adaptive software for embedded systems,
context-oriented programming, and system-level adaptiveness in
CPSs. We briefly survey paradigmatic examples in the respective fields
and how they relate to our work.

Works in self-adaptive embedded system software
include a vast array of contributions from requirement engineering to
verification~\cite{cheng:adaptive}. Co-design approaches also exist
where the hardware/software boundaries blur for greater
flexibility~\cite{diguet11:closed}. Unlike in our work, several of
these solutions focus on at most a few environmental dimensions, each
requiring sophisticated self-adaptive functionality. In our target
applications, complexity arises especially from the several
combinations that multiple environmental dimensions concurrently
generate. Nevertheless, most of the existing solutions would be hardly
applicable in resource-constrained CPSs, due to a significant
run-time overhead.

The COP model~\cite{Hirschfeld08} is implemented in several
high-level
languages~\cite{Kamina11,Ghezzi10,Bardram05,Sehic11,Salvaneschi12}. These
are, however, generally unfeasible on the platforms we target. We
borrow a few of these concepts; for example, our layered functions in
context groups are akin to the layer-in-class by Salvaneschi et
al.~\cite{Salvaneschi12}, and adapt them to the limitations of
low-level programming platforms. In this area, CPSs are rather seen as
providers of raw sensor data. For example, Sehic et al.~\cite{Sehic11}
design context-aware extensions to Java where input data comes from
wireless sensor networks. Differently, we bring COP down to the
resource-constrained CPS nodes, enabling self-adaptive software right
on the devices that directly interact with the surrounding
environment.

Specific cases of run-time adaptation are seen at system level in the
CPS literature. For example, Steine et al.~\cite{Steine11} focus on
run-time reconfiguration of wireless sensor networks by dynamically
adjusting protocol values depending on environmental conditions. MAC
and routing protocols expressly designed with self-adaptive
functionality are also found in the existing
literature~\cite{Park08,Bourdenas11}. Such efforts essentially address
a different, yet complementary problem compared to ours. While we aim
to provide design-time and programming support to implement
self-adaptive software in this domain, these works focus on the actual
problem-specific adaptation logic.


% There are some attempts to apply context-oriented programming for the software
% for CPSs. Sehic et al.~\cite{Sehic11}, for example, proposed a macro-language
% for rapid development of context-aware applications for CPSs. Wood et
% al.~\cite{Wood08} developed a CPS for Assisted-Living and Residential
% Monitoring, which utilizes a technique of a context-depended behavior. Another
% work, provided by Ghezzi et al.~\cite{}, is devoted to the language
% support for bringing a context-awareness for CPSs.

% Several attempts were also performed towards self-adaptive software for CPSs. In
% the work of Steine et al.~\cite{Steine11} authors were focused on run-time
% reconfiguration of WSNs by adjusting protocol values dynamically depending on
% the conditions the system operates in. Marques et al.~\cite{Marques11} propose a
% run-time evaluation of QoS characteristics for better adaptation. Another
% research was provided towards self-adaptive protocols: MAC-protocol by Park et
% al.~\cite{Park08} and routing protocol by Bourdenas et al.~\cite{Bourdenas11}.
% Self-adaptivity is also considered as a main ingredient of run-time error
% handling~\cite{Bourdenas10} and energy management~\cite{Jiang07}.

% Despite the research activity in the area of self-adaptive software for CPSs is
% quite intensive, it remains fragmented, and no holistic approach yet developed.
% Current research has two extreme directions:~\emph{i)} high-level software,
% which brings adaptivity only in application level using CPSs as raw-data
% providers~\cite{Wood08,Sehic11,Ghezzi10}, and~\emph{ii)} low-level software,
% which is focused on a system level ignoring an application
% level~\cite{Steine11,Marques11,Park08,Bourdenas11,Bourdenas10,Jiang07}.
% We combine different aspects of self-adaptivity for CPSs and provide robust
% approach for self-adaptivity on both system and application levels, which makes
% our work novel in this field.

% Salvaneschi et al.~\cite{SalvaneschiTBP}\footnote{to be published}\cite{Salvaneschi12} 
% outlined possible techniques for language adaptation for context-oriented
% programming. We not only utilize concepts of COP, but also provide a novel approach for
% bringing context-oriented programming to CPSs. To this end, we modified and
% applied such techniques as: \emph{Per-Object activation} and
% \emph{Layer-in-Class} \cite{Salvaneschi12}. The latter implies the
% implementations of layers~\cite{Costanza05} -- i.e. behavioral variations -- in
% one class. In our approach we used \emph{context groups} to aggregate behavioral
% variations, which are invoked by explicit activation of corresponding
% \emph{context}, i.e. we adopted \emph{Per-Object activation}, where an
% \emph{object} is a~\emph{context}.


%%% Local Variables: 
%%% mode: latex
%%% TeX-master: "paper"
%%% End: 
