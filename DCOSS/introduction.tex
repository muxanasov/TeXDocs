\newsavebox{\boxnescEX}
\Savebox{\boxnescEX}{
\hspace{-0.25cm}
\begin{minipage}[l]{\columnwidth}
\begin{lstlisting}[style=conescframe]
module ReportLogs {
  uses interface Collection;
  uses interface DataStore;  
}implementation {
 int base_station_reachable = 0; 
*\lstnote{detect1}* event msg_t Beacon.receive(msg_t msg) {
*\lstnote{check1}*  if (!acceleromenter_detects_activity()) 
    return;
  if (call Battery.energy() <= THRESHOLD)
*\lstnote{check2}*    return; 
  base_station_reachable = 1;
  call GPS.stop()
  call BaseStationReset.stop(); 
  call BaseStationReset.startOneShot(TIMEOUT);}
 event void BaseStationReset.fired() {
*\lstnote{detect2}*  base_station_reachable = 0;}
*\lstnote{adapt1}* event void ReportPeriod.fired() {
  switch (base_station_reachable){
   case 0:
    call DataStore.deposit(msg);
   case 1:
*\lstnote{adapt2}*    call Collection.send(msg);}}}
\end{lstlisting}
\end{minipage}
}
%%% Local Variables: 
%%% mode: latex
%%% TeX-master: "bare_conf"
%%% End: 
\section{Introduction} 

Programmers design and implement Wireless Sensor Networks (WSN)
software to enable interactions with the real world at unprecedented
granularity and scale. As such, WSN software is continuously confronted
with a range of largely unpredictable environment dynamics and
changing requirements, besides the obvious resource constraints. Such
state of affairs demands WSN software to \emph{adapt} to a range of
different situations. Notwithstanding the advances in WSN
programming~\cite{mottola10:survey}, however, programmers are sorely
missing dedicated support to realize adaptive WSN
software.

\fakepar{Example application} Consider a wildlife tracking
application~\cite{pasztor10:selective}. Sensor nodes are embedded in
collars attached to animals to study their social interactions. The
nodes are equipped with sensors to track an animal's movement, e.g.,
using GPS and accelerometers, and to detect its health conditions,
e.g., based on body temperature. Small solar panels harvest energy
from the environment to prolong a node's lifetime. A low-power
short-range radio allows the nodes to discover each other based on
periodic radio beaconing.  A node logs the radio contacts to track an
animal's encounters with other animals. The radio is also used to
off-load the contact traces when in reach of a fixed base-station.

The nodes run on batteries, making energy a precious resource that
developers may need to trade against the system's functionality,
depending on the situation. For example, sensor sampling consumes
non-negligible energy, especially for devices such as the
GPS. Depending on the desired granularity and on the difference
between consecutive GPS readings---the latter taken as indication of
the pace of movement---developers may tune the GPS sampling frequency
accordingly. The contact traces can be sent directly to the
base-station whenever in reach, but they need to be stored locally
otherwise.  When the battery is running low, developers may turn the
GPS sensor off to make the node survive until the next encounter with
a base-station, not to lose the collected contact traces\lm{Changes to
  the pic required.}.

\fakepar{Problem} Taking into explicit account every possible
situation in the design and implementation of WSN software is a
challenge. Critically, \emph{multiple combined dimensions}
concurrently determine how the software should adapt its operation,
e.g., battery levels and physical locations in our example
application. Using available approaches, this typically results in
entangled implementations that are difficult to debug, to maintain,
and to evolve.  As the number of dimensions affecting the execution
(and their combinations) grows, the implementations quickly turn into
``spaghetti code''~\cite{finne10improving}.

\putsnippet{
 caption=Example nesC implementation of adaptive functionality: several orthogonal functionality are entangled and need to share global data.,
 label=fig:nesc,
 boxname=boxnescEX
}

Fig.~\ref{fig:nesc} shows an intuitive, yet greatly simplified
example, using nesC~\cite{gay03nesc}. The code implements the behavior needed
in wildlife tracking to send contact logs to the base-station whenever
reachable, or to store them locally otherwise. Several orthogonal
concerns become intertwined and dependent on each other. For example,
determining what operating mode to apply---implemented in
line~\lstref{detect1} to~\lstref{detect2}---rests within the same
module as the actual adaptive processing---implemented in
line~\lstref{adapt1} to~\lstref{adapt2}. Indeed, the two codes need to
share global data, in this case, the \code{base\_station\_reachable}
flag. Managing such global state rests entirely on the programmers'
shoulders. Moreover, the checks to apply before changing operating
mode, such as those on line~\lstref{check1} to~\lstref{check2}, appear
interleaved with the change of mode itself. Finally, the specific
implementation of adaptive functionality---being either
\code{DataStore} or \code{Collection}---is visible from the caller
module, further coupling the two.\lm{The line numbers are huge when
  made of 2 digits... :)}

In such a situation, debugging, maintaining, and evolving the
implementations is going to be difficult. Modifying the code in one
place would likely require changes in several others. Alternative nesC
implementations of the functionality in Fig.~\ref{fig:nesc} are of
course possible to partly ameliorate the problem. However, qualitative
evidence gathered by looking at publicly available implementations,
e.g., within the TinyOS codebase~\cite{tinyos}, and our own experience
with real deployments~\cite{ceriotti11:light,ceriotti09:monitoring}
indicate that similar implementation patterns are indeed very common.

% \fakepar{Context-oriented programming (COP)}

\fakepar{Contribution and road-map} We aim to redress this state of
affairs by enabling a notion of Context-Oriented Programming
(COP)~\cite{} in WSN software. COP fosters a strict separation of
concerns in implementing adaptive software. This is achieved through
two key notions: \emph{i)} the different situations the software needs
to operate in are mapped to different \emph{contexts}, and \emph{ii)}
the different context-dependent behaviors are encapsulated in so
called \emph{layered functions}, that is, functions whose behavior
changes---transparently to the caller---depending on context.

COP already proved its effectiveness in creating context-aware
mainstream software, such as graphical user interfaces~\cite{Keays03}
and text editors~\cite{Kamina11}, using COP extensions of popular
high-level languages~\cite{Salvaneschi12}. At present, however, COP
remains a far cry from being applicable to WSNs. The severe resource
constraints that limit the functionality attainable with existing WSN
programming languages, for example, the inability to create run-time
instances of components, prevents applying COP in WSN programming as
is.

To address this issue, we borrow the key concepts of contexts and
layered functions from COP and design context-oriented programming
abstractions for WSN software. To this end,
Section~\ref{sec:appdesign} illustrates design concepts for adaptive
WSN software conceived to remain independent of a specific programming
language. In doing so, our goal is to decouple the concepts, that are,
the abstractions, from their realization in a concrete language, thus
facilitating their application to multiple WSN languages. One such
realization is \conesc, our own COP extension to nesC, described in
Section~\ref{sec:conesc}. We choose nesC as the target language in
that, besides its widespread adoption, it fosters a node-centric
view. Indeed, we argue that in most WSN applications, adaptation
decisions are local, and rarely affect the system as a
whole. Section~\ref{sec:conesc} demonstrates how \conesc renders the
processing in Fig.~\ref{fig:nesc} nicely decoupled in different
modules, and hence easier to debug and to evolve.

We implement a dedicated translator, described in
Section~\ref{sec:translator}, that converts \conesc code to pure
nesC. The latter source code is fed as input to the standard nesC
toolchain to obtain the binary to deploy onto the WSN nodes. Based on
three representative applications, the results we illustrate in
Section~\ref{sec:eval} indicate that \conesc implementations are
increasingly decoupled and distinctively simpler. For example, the
automatic analysis we perform with a model-checking tool to measure
the number of per-function states that programmers need to deal with
shows a 50\% reduction in favor of \conesc, indicating that the latter
implementations are mot likely easier to debug and to
maintain. Crucially, these advantages come at a very modest price in
terms of system overhead: the MCU overhead when performing calls to
layered functions is negligible, while we measure a a maximum 2.5\%
(4.5\%) overhead in program (data) memory in our test applications.

We conclude the paper by surveying related efforts in
Section~\ref{sec:related}, and with brief concluding remarks in
Section~\ref{sec:ending}.


% We are claiming that the notion of \emph{context} in a low-level software will
% significantly improve readability and re-usability of the code, as well as
% debugging and maintaining of the application. We also believe this notion will
% allow a developer to represent any dynamic variability. Moreover, our approach
% supports combinations of adaptation, which are usually taking a place in the
% real world.

% Consider for example a sensor network for wildlife tracking~\cite{pasztor10},
% which is intended to track animals' movements, social interactions and health
% conditions. To this end, nodes are attached to collars of animals and equipped
% with GPS and accelerometer sensors to track the movements, temperature sensor
% to monitor health conditions, and low-power short-range radio to perform a
% periodic beaconing. The latter allows the nodes to discover each other and then
% log the radio contacts to study an animal's social interactions.

% The way to create this kind of self-adaptive software is described in
% Sec.~\ref{sec:cop}, but the development would be difficult not only because of
% several adaptation dimensions, but also due to extreme resource limitations --
% the main challenge in WSNs. By addressing to these issues we introduce \conesc
% -- a context oriented extension of nesC language described in
% Sec.~\ref{sec:conesc}. To test our approach we developed a tool described in
% Sec.~\ref{sec:translator} to translate \conesc sources into nesC language. We
% discuss then our evaluation results in Sec.~\ref{sec:eval} in details.

%%% Local Variables: 
%%% mode: latex
%%% TeX-master: "bare_conf"
%%% End: 
