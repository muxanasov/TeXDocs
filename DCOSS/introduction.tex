\section{Introduction} 

Wireless Sensor Networks are the core of communication between the world and
machine. They are used to monitor the environment and gather data from it, what
makes them inherently environment dependent. Due to unpredicted dynamics in
different dimensions, software for WSNs should adapt to it at run-time. Without
proper language support the implementation of this kind of software is quite
complex and cumbersome. The adaptation, thus, would be achieved by using
conditional statements, what leads to the software that is tangled to the
possible adaptation.

We are claiming that the notion of \emph{context} in a low-level software will
significantly improve readability and re-usability of the code, as well as
debugging and maintaining of the application. We also believe this notion will
allow a developer to represent any dynamic variability. Moreover, our approach
supports combinations of adaptation, which are usually taking a place in the
real world.

Consider for example a sensor network for wildlife tracking~\cite{pasztor10},
which is intended to track animals' movements, social interactions and health
conditions. To this end, nodes are attached to collars of animals and equipped
with GPS and accelerometer sensors to track the movements, temperature sensor
to monitor health conditions, and low-power short-range radio to perform a
periodic beaconing. The latter allows the nodes to discover each other and then
log the radio contacts to study an animal's social interactions.

The way to create this kind of self-adaptive software is described in
Sec.~\ref{sec:cop}, but the development would be difficult not only because of
several adaptation dimensions, but also due to extreme resource limitations --
the main challenge in WSNs. By addressing to these issues we introduce \conesc
-- a context oriented extension of nesC language described in
Sec.~\ref{sec:conesc}. To test our approach we developed a tool described in
Sec.~\ref{sec:translator} to translate \conesc sources into nesC language. We
discuss then our evaluation results in Sec.~\ref{sec:eval} in details.

\hrule In doing so, we advocate that higher levels of abstractions,
such as macroprogramming ones, do not provide sufficient visibility
into the system and environment state. Adaptation decisions are local,
the system rarely adapts as a whole.
