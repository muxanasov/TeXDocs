\section{Introduction} To mention: The main goal of Wireless Sensor Networks
(WSNs) is to monitor the environment, thus WSNs are inherently environment
dependent. It leads to a number of situations and combinations of situations,
which programmer should be care of, and because of that applications for WSNs
are usually very complex. It is hard to read, reuse, debug and maintain this
kind of applications.

We are claiming that the notion of Context in a low-level software will
significantly improve these aspects (readability and re-usability of the code
as well as debugging and maintaining of the application).

Since WSNs are usually built on embedded platforms, there are CPU, memory,
energy consumption and even language restrictions.

...

Consider for example a sensor network for wildlife tracking~\cite{pasztor10},
which is intended to track animals' movements, social interactions and health
conditions. To this end, nodes are attached to collars of animals and equipped
with GPS and accelerometer sensors to track the movements, temperature sensor
to monitor health conditions, and low-power short-range radio to perform a
periodic beaconing. The latter allows the nodes to discover each other and then
log the radio contacts to study an animal's social interactions.

To mention: structure of the paper.

\hrule In doing so, we advocate that higher levels of abstractions,
such as macroprogramming ones, do not provide sufficient visibility
into the system and environment state. Adaptation decisions are local,
the system rarely adapts as a whole.
