\section{Conclusion}\label{sec:ending}

We presented programming abstractions for implementing adaptive WSN software. 

While the area of adaptive is Due to absence of dedicated support of developing adaptive applications.

By bringing COP down to WSN devices, we provide language independent design concepts to organize

The conclusion goes here.

To mention: While we have negligible CPU and memory overhead, the code is more decoupled and each module is more isolated.

As we observed in~\ref{sec:eval} based on three representative applications, our own context-oriented extension of nesC, called \conesc, simplifies the developing for WSNs. We shown that along with decoupling of software components, we gain also a \approx50\% reduction in the number of per-function states that programmers need to deal with. Moreover, due to decoupled modular structure, \conesc applications can be evolved with lesser efforts as compared to their nesC counterparts, as we shown in Sec.~\ref{sec:evolve}. The price is, however, negligible: we observed the overhead in memory (data) of maximum 2.5\% (4.5\%), while the maximum of MCU overhead for context transition (function call) is only 28 (4) MCU cycles.

To our knowledge, efforts related to ours can be divided into WSN-specific adaptation and programmer support for adaptation outside WSN. \conec brings the 


% conference papers do not normally have an appendix


% use section* for acknowledgement
\section*{Acknowledgment}


The authors would like to thank...





% trigger a \newpage just before the given reference
% number - used to balance the columns on the last page
% adjust value as needed - may need to be readjusted if
% the document is modified later
%\IEEEtriggeratref{8}
% The "triggered" command can be changed if desired:
%\IEEEtriggercmd{\enlargethispage{-5in}}

% references section

% can use a bibliography generated by BibTeX as a .bbl file
% BibTeX documentation can be easily obtained at:
% http://www.ctan.org/tex-archive/biblio/bibtex/contrib/doc/
% The IEEEtran BibTeX style support page is at:
% http://www.michaelshell.org/tex/ieeetran/bibtex/
%\bibliographystyle{IEEEtran}
% argument is your BibTeX string definitions and bibliography database(s)
%\bibliography{IEEEabrv,../bib/paper}
%
% <OR> manually copy in the resultant .bbl file
% set second argument of \begin to the number of references
% (used to reserve space for the reference number labels box)