\section{Conclusion}\label{sec:ending}

We presented programming abstractions for implementing adaptive WSN
software. By borrowing from COP, we conceived language-independent
design concepts mirrored in a concrete language
implementation---\conesc---that extends nesC with COP constructs. Our
dedicated translator converts \conesc code to plain nesC code, then
handed over to the standard nesC toolchain. Based on three
representative applications, we observed that \conesc greatly
simplifies developing adaptive WSN software. For example, we found
that, along with increased decoupling of software components, we gain
a $\approx$50\% reduction in the number of per-function states that
programmers need to deal with, and applications can be evolved with
reduced efforts compared to their nesC counterparts. The price for
gaining such advantages is, however, negligible: we observed an
overhead of 2.5\% (4.5\%) in program (data) memory, whereas the MCU
overhead is negligible. The \conesc toolchain is available at
\url{code.google.com/p/conesc}.

% To our knowledge, efforts related to ours can be divided into WSN-specific
% adaptation and programmer support for adaptation outside WSN. Meanwhile, \conesc
% brings the design-time programming support for adaptive low-level WSN software.

% use section* for acknowledgement
%\section*{Acknowledgment}
%The authors would like to thank...


%%% Local Variables: 
%%% mode: latex
%%% TeX-master: "bare_conf"
%%% End: 
