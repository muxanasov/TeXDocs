\newsavebox{\boxcc}
\Savebox{\boxcc}{
\hspace{-0.25cm}
\begin{minipage}[l]{\columnwidth}
\begin{lstlisting}[style=conescframe]
context group BaseStationG {
*\lstnote{layereddef}* layered command void report(msg_t msg);
}implementation {
*\lstnote{contexts}* contexts Reachable,
*\lstnote{isdefault}*          Unreachable is default,
*\lstnote{iserror}*          ErrorC is error;
 components Routing, Logging, LedsC;
 Reachable.Collection -> Routing;
 Unreachable.DataStore -> Logging;
 ErrorC.Leds -> LedsC;}
\end{lstlisting}
\end{minipage}
}
\newsavebox{\boxbscm}
\Savebox{\boxbscm}{
\hspace{-0.25cm}
\begin{minipage}[l]{\columnwidth}
\begin{lstlisting}[style=conescframe]
module BaseStationContextManager {
*\lstnote{cgdecl}* uses context group BaseStationG;
}implementation {
 event msg_t Radio.receive(msg_t msg) {
*\lstnote{actBS}*  activate BaseStationG.Reachable;
  call BSReset.startOneShot(INTERVAL);}
 event void BSReset.fired() {
*\lstnote{actNoBS}*  activate BaseStationG.Unreachable;}}
\end{lstlisting}
\end{minipage}
}
\newsavebox{\boxc}
\Savebox{\boxc}{
\hspace{-0.25cm}
\begin{minipage}[l]{\columnwidth}
\begin{lstlisting}[style=conescframe]
context Unreachable {
*\lstnote{dependence}* transitions InRange iff ActivityG.Running;
 uses interface DataStore;
}implementation {
*\lstnote{activated}* event void activated(){//...}
*\lstnote{deactivated}* event void deactivated(){//...}
*\lstnote{layeredimp}* layered command void report(msg_t msg){
  call DataStore.deposit(msg);}}
\end{lstlisting}
\end{minipage}
}
\newsavebox{\boxirc}
\Savebox{\boxirc}{
\hspace{-0.25cm}
\begin{minipage}[l]{\columnwidth}
\begin{lstlisting}[style=conescframe]
context Reachable {
 uses interface Collection;
 uses context group BatteryG; 
}implementation {
*\lstnote{check}* command bool check(){
  return call BatteryG.getContext() == BatteryG.Normal;}
*\lstnote{layeredimp2}* layered command void report(msg_t msg){
  call Collection.send(msg);}}
\end{lstlisting}
\end{minipage}
}
\newsavebox{\boxlc}
\Savebox{\boxlc}{
\hspace{-0.25cm}
\begin{minipage}[l]{\columnwidth}
\begin{lstlisting}[style=conescframe]
context Low {
*\lstnote{triggers}* triggers BaseStationG.OutRange;
}implementation {//...}
\end{lstlisting}
\end{minipage}
}
\newsavebox{\boxmc}
\Savebox{\boxmc}{
\hspace{-0.25cm}
\begin{minipage}[l]{\columnwidth}
\begin{lstlisting}[style=conescframe]
configuration ApplicationC {
}implementation {
*\lstnote{declaration}* components BaseStationG, Application;
 //...
*\lstnote{wiring}* Application.BaseStationG -> BaseStationG;}
\end{lstlisting}
\end{minipage}
}
\newsavebox{\boxmm}
\Savebox{\boxmm}{
\hspace{-0.25cm}
\begin{minipage}[l]{\columnwidth}
\begin{lstlisting}[style=conescframe]
module Application {
*\lstnote{cgdecl}* uses context group BaseStationG;
}implementation {
 event void Timer.fired() {
*\lstnote{calling}*  call BaseStationG.report(msg);}
*\lstnote{eventCC}* event void BaseStationG.contextChanged(context_t con) {
*\lstnote{concheck}*  if(con == BaseStationG.Reachable)
   dbg("Debug", "Base station is reachable!");}}
\end{lstlisting}
\end{minipage}
}
\newsavebox{\boxnmc}
\Savebox{\boxnmc}{
\hspace{-0.25cm}
\begin{minipage}[l]{\columnwidth}
\begin{lstlisting}[style=conescframe]
context NotMoving {
*\lstnote{transitions}* transitions Resting;
}implementation {//...}
\end{lstlisting}
\end{minipage}
}
\section{ConesC}\label{sec:conesc}

% \conesc provides a design-time support for developing a self-adaptive software,
% which alternates its behavior at run-time including catching and handling
% errors. Along with this, our approach provides a deep modularization, since
% behavioral variations are encapsulated into different modules. The latter, as we
% show in Sec.~\ref{sec:evalcomp}, are highly decoupled, which enhances code
% readability and re-usability as well as debugging and developing processes.

We illustrate how we render the concepts in
Section~\ref{sec:appdesign} within \conesc: our own context-oriented
extension to nesC.  We describe a notion of context module and
configuration in Section~\ref{subsec:components}, and discuss in
Section~\ref{subsec:usage} how programmers use these constructs to
specify an application's adaptive behavior.
Section~\ref{subsec:rules} describes how \conesc programmers deal with
context transitions and their relations.

\subsection{Context Group and Individual Contexts}\label{subsec:components}

\putsnippet{
 caption=Context group in \conesc.,
 label=fig:ccc,
 boxname=boxcc
}

Context groups in \conesc extend nesC configurations. Programmers use
context groups to declare layered functions and the contexts providing
the corresponding behavioral variations. Fig.~\ref{fig:ccc} shows an
example for the \emph{Base-station} group. A layered \code{report}
function is declared on line~\lstref{layereddef} by using the keyword
\code{layered}. The contexts providing the necessary behavioral
variations are specified following the keyword \code{contexts} on
line~\lstref{contexts}. In this case, programmers define two such
contexts, depending on base-station reachability. The \code{is
  default} modifier, shown on line~\lstref{isdefault}, indicates what
context is active at start-up. The next \code{is error} modifier on
line~\lstref{iserror} declares context \code{MyErrorC} as an
\emph{error} context, which programmers may optionally use to handle
errors during the execution, as we discuss in
Section~\ref{subsec:rules}. If an error context is not declared, it is
generated automatically.

\putsnippet{
 caption=\emph{Reachable} context.,
 label=fig:irc,
 boxname=boxirc
}

\putsnippet{
 caption=\emph{Unreachable} context.,
 label=fig:cc,
 boxname=boxc
}

The individual contexts in \conesc extend the standard nesC modules by
providing context-dependent implementations of layered function
declared in context groups. Only one context at a time can be
\emph{active} in a group to provide an implementation for the given
layered functions. For example, Fig.~\ref{fig:irc} and~\ref{fig:cc}
show \conesc snippets for the \emph{Reachable} and \emph{Unreachable}
contexts of Fig.~\ref{fig:ccc}. They provide different implementations
for \code{report} depending on the situation. If the base-station is
\emph{Reachable}, and thus the corresponding context is active, the
code transmits the message to the base-station, as in
line~\lstref{layeredimp2} of Fig~\ref{fig:irc}. Differently, the code
deposits a message in local memory as in line~\lstref{layeredimp} of
Fig.~\ref{fig:cc}.

Programmers can specify operations upon activating a context, such as
initializing variables or enabling/disabling hardware modules. For
example, on entering the \emph{Reachable} context, programmers may
decide to disable the GPS sensor, as location information can be
inferred from the (static) base-station. Programmers specify this
functionality within the body of a predefined \code{activated} event,
as in line~\lstref{activatedUnreachable} of
Fig.~\ref{fig:irc}. Similarly, programmers may specify clean-up
operations within \code{deactivated} events, as in
line~\lstref{deactivated} of Fig.~\ref{fig:irc}.  Providing
an implementation for these events, however, is not mandatory.

\subsection{Execution}\label{subsec:usage}

% Context groups encaspulate behavioral variations sharing some common
% characteristic or related to the same funcitonality. Within the
% \emph{Base-station} group of Fig.~\ref{fig:ccc}, for example,
% programmers activate different behavioral variations for function
% \code{report} depending on base-station reachability. 

\putsnippet{
 caption=Base-station context manager.,
 label=fig:bscm,
 boxname=boxbscm
}

Fig.~\ref{fig:bscm} shows a sample snippet of code to detect and to
activate the proper context in the base-station example. Programmers
can, anywhere in the code, trigger explicit transitions between contexts
in a group. This is as simple as using the \code{activate} keyword
followed by a full context name. In this fragment of code, the
\emph{Reachable} context is activated on line~\lstref{actBS} as soon
as a beacon from the base station is received. Should the timeout
expire with no more beacons received, context \emph{Unreachable} is
activated on line~\lstref{actNoBS}. Either context change results in a
different context-dependent implementation of \code{report} to be
activated.

\putsnippet{
 caption=Caller module.,
 label=fig:mm,
 boxname=boxmm
}

Modules using layered functions perform function calls transparently
w.r.t.\ the available contexts and, most importantly, independently of
what context is active at a given moment. Fig.~\ref{fig:mm} shows one
such example for function \code{report}. Following the indication that
context group \emph{BaseStationG} is used, as specified on
line~\lstref{cgdecl}, the call to the layered function \code{report}
does not refer to the individual contexts. The net advantage is that
the use of context-dependent functionality is fully decoupled w.r.t.\
context detection and activation. The two may be implemented even in
different modules.


Should programmers of caller modules need to find out
about context changes, a predefined event~\code{contextChanged} is
fired corresponding to every context change, as in
line~\lstref{eventCC} in Fig.~\ref{fig:mm}. % This event can be caught
% and handled, as it is shown on the line~\lstref{eventCC} in
% Fig.~\ref{fig:mm}, but it is not mandatory though.
Within the event handler, programmers can access constant values that
our translator automatically generates to find out what context was
activated and to react accordingly, as shown on
line~\lstref{concheck}.


% This section shows how ConesC can be used to invoke a behavioral
% variation. Here we describe one aspect of an application's behavioral variation,
% which is related to the Base Station availability. As it was mentioned before,
% if a node receives a beacon from the Base Station it activates the \emph{Reachable}
% context, and the \emph{Unreachable} context in case of timeout.
% Since the behavioral variation is encapsulated in a context group \emph{BaseStationG},
% a developer may not care about the context implementation and use a context group
% Tiny fix.to change a behavior of the application at run-time. Fig.~\ref{fig:mc} depicts the main
% configuration, where the base station group is declared on line~\lstref{declaration} and wired
% as a standard component afterward on the line~\lstref{wiring}.

% \putsnippet{
%  caption=Main configuration.,
%  label=fig:mc,
%  boxname=boxmc
% }

\subsection{Transition Rules}\label{subsec:rules}

\putfigure{caption=Context activation rules.,label=fig:ad}{
 \centering
 \includegraphics[width=\columnwidth]{pdf/activation_diagram}
}

In general, programmers need to take significant care of context
transitions, in that the latter may drastically change an
application's behavior. To better support programmers in doing so,
every context transition in \conesc entails several checking stages,
as shown in Fig.~\ref{fig:ad}. A successful check allows the
transition to continue, while the failure leads either to the
canceling of the transition or to activation of the \emph{Error}
context. % Our translator automatically generates this context unless a
% programmer-provided one is explicitly declared in a context group by
% using the keyword \code{is error}, as shown on line~\lstref{iserror}
% in Fig.~\ref{fig:ccc}.

The first check in Fig.~\ref{fig:ad} looks at feasible transitions. In
the context diagram of Fig.~\ref{fig:wtd}, within the \emph{Activity}
group, it is only possible to transition from \emph{NotMoving} to
\emph{Resting}. Feasible transitions are specified within the
individual contexts using the keyword \code{transitions} as in
line~\lstref{transitions} of Fig.~\ref{fig:nmc}. An attempt to
initiate a transition from a context to one that is not
explicitly listed in the former leads to the activation of the
\emph{Error} context. Indeed, such occurrences typically represent a
significant design or implementation flaw requiring special handling
at run-time, which programmers implement within the \emph{Error} context.

\putsnippet{
 caption=\emph{NotMoving} context.,
 label=fig:nmc,
 boxname=boxnmc
}

There may also exist relations across context groups. For example,
within the \emph{Base-station} group, a transition from
\emph{Unreachable} to \emph{Reachable} is likely only meaningful if
context \emph{Running} within the \emph{Activity} group is active,
indicating the animal was actually moving when the node gained
base-station connectivity. These inter-group relations are covered in
our design by context dependencies, declared as shown on
line~\lstref{dependence} in Fig.~\ref{fig:cc}. Within the
\code{transitions} clause, the keyword \code{iff} is optionally
employed to indicate the full name of another context whose activation
is a requisite to perform the given transition. The second check in
Fig.~\ref{fig:ad} verifies this rule, again leading to the
\emph{Error} context in case of violations, giving programmers a
chance to handle the situation.

The last check in Fig.~\ref{fig:ad} considers violations to ``soft''
requirements that do not necessarily indicate a design or
implementation flaw.  For example, before activating the
\emph{Reachable} context, programmers may want to check that
sufficient energy is available to invest in bulk data transfers to the
base-station. Should this not be the case, they may defer the
activation of the \emph{Reachable} context until the solar panels
gather sufficient energy. To implement such processing, \conesc
programmers specify the proper conditions in the body of a predefined
\code{check} command, as shown in line~\lstref{check} of
Fig.~\ref{fig:irc}. If \code{check} returns false, the initiated
context transition does not occur, and the system remains in the
previous context.

\putsnippet{
 caption=\emph{Low} context.,
 label=fig:lc,
 boxname=boxlc
}

Dually, programmers may need to proactively initiate context
transitions as the result of other contexts being activated. The
scenario is symmetric to the previous one: if the base-station is
\emph{Reachable}, but a context transition is initiated to context
\emph{Low} in the \emph{Battery} group of Fig.~\ref{fig:wtd}, the
available energy is running low and it is probably better to refrain
from radio communications. This makes sure the node does not
completely turn off before the solar panels re-gain energy. Our design
allows programmers to express this processing by using the
\code{triggers} keyword, as shown on line~\lstref{triggers} in
Fig.~\ref{fig:lc}. The \code{triggers} keyword points to a context
that is to be activated as the result of the enclosing context being
activated. The same checks shown in Fig.~\ref{fig:ad} apply to this
type of transitions.


% Other types of inter-group relations imply automatic triggering of
% context transition. Considering \emph{Battery} group in our example,
% we can notice that for further energy saving developers may want to
% trigger a transition to \emph{Unreachable} context within the
% \emph{Base Station} group as long as \emph{Low} context is active.



%%% Local Variables: 
%%% mode: latex
%%% TeX-master: "bare_conf"
%%% End: 
