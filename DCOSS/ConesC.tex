\section{Context Oriented nesC (ConesC)}
To mention: detailed description of ConesC, using Wildlife tracking application.

Programming in ConesC is tightly coupled with environment-dependent behavior of the application and exploits two main concepts: \emph{i)} individual contexts and their transitions and \emph{ii)} context groups. The former represent different environment situations where a system may find itself. Each context maps a separate state of application-level or system-level. As the environment conditions change, a programmer initiates a context transition by \emph{activating} corresponding context. Context group is a set of contexts sharing common characteristics. Thus, whenever a transition between involved contexts occurs it is determined by changes in the same physical quantity.