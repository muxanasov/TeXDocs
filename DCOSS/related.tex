\section{Related Work}\label{sec:related}

Efforts related to ours are roughly divided in two categories. On one
hand, as the need for adaptivity in WSNs was immediately recognized
because of the intimate environment interactions, several system-level
solutions exist to provide adaptive behaviors at different levels in
the stack. Our work is complementary to these efforts: rather than
devising problem-specific adaptation mechanisms, we present design
concepts and programming constructs to facilitate the implementation
of such mechanisms. On the other hand, programming support for
adaptation, including the application of COP, is more extensively
studied for more traditional computing platforms, missing, however, a
dedicated port of concepts and abstractions to resource-constrained
devices. In the following, we briefly survey the literature based on
examples closer to our work.

\fakepar{WSN-specific adaptation} Solutions in this category often
target run-time adaptation of MAC and routing protocols. For example,
Zimmerling et al.~\cite{zimmerling12} focus on an adaptation of MAC
protocol parameters depending on link qualities, topology dynamics,
and traffic loads. Based on an mathematical formulation of the problem
at hand, the base-station computes optimized MAC parameters to satisfy
user-provided performance goals. Another example is that of Bourdenas
et al.~\cite{Bourdenas11}, who design a routing protocol with
dedicated adaptive functionality. Using a custom forecasting approach,
the system can predict the conditions of the network and anticipate
the changes required in the routing protocol behavior. Our work is
intended to serve the needs of those needing to implement such
adaptive functionality, easing the implementation chore.

Closer to our goals are the works on self-organizing WSN
architectures. For example, Subramanian and Katz~\cite{Subramanian00},
define a component model to build adaptive WSN architectures. Their
work, however, is again intended for specific adaptation needs,
that are, those arising in static WSNs deployed for large-scale sensing
tasks. For example, the work is not applicable in mobile scenarios akin to a
wildlife tracking application. Diguet et
al.~\cite{Diguet11} blur the boundaries between software and hardware
to gain additional flexibility in providing adaptive
functionality. Their design, however, leads to application-specific
implementations. Our work aims to be more general than these efforts,
as we demonstrated by applying \conesc to diverse application
scenarios, as discussed in Section~\ref{sec:eval}.

% Over decade many contributions are addressing to the self-adaptivity of WSNs in
% both hardware and software perspectives. Unlike in our work, they are mostly
% focusing on a few environmental dimensions, while the adaptation rarely occurs
% individually. Moreover, the solutions address the actual problem-specific
% adaptation logic, while our approach provides abstractions to design
% self-adaptive software regardless the problem.

\fakepar{Programming support for adaptation} Some works explicitly
provide programming support for adaptation \emph{outside} the WSN,
essentially regarding the latter as an application-agnostic source of
raw data. For example, Sehic et al.~\cite{Sehic11} design a Java-based
framework for context-aware applications using input data from a
WSN. Differently, we bring a notion of context down to the
resource-constrained devices, allowing to implement context-aware
behaviors right on the WSN device.

A natural way to handle adaptivity at the programming level is to
embed some notion of COP within an existing language, similar to what
we do. Indeed, several high-level languages already feature COP
extensions~\cite{Kamina11,Salvaneschi12,Sehic11}.  Such approaches,
however, are far from being applicable in WSNs, due to specific
application requirements and resource limitations. For example, the
multiple dimensions of adaptive behavior germane to WSN applications
are rarely considered in existing works. In \conesc, we borrow
concepts from COP and adapt them to the typical requirements arising
in WSN applications and to the limitations of related platforms.

% These efforts are solving different aspects of the same problem -- effective
% adaptivity of WSNs -- that we address in this work. The complexity of the
% applications we target arises from the different environmental dimensions the
% software should adapt to. We also overcome resource limitations to bring the
% adaptation on the low-level programming platforms.

%%% Local Variables: 
%%% mode: latex
%%% TeX-master: "bare_conf"
%%% End: 

