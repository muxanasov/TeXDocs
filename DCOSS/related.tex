\section{Related Work}\label{sec:related}

Over decade many contributions are addressing to the self-adaptivity of WSNs in
both hardware and software perspectives. Unlike in our work, they are mostly
focusing on a few environmental dimensions, while the adaptation rarely occurs
individually. Moreover, the solutions address the actual problem-specific
adaptation logic, while our approach provides abstractions to design
self-adaptive software regardless the problem.

The self-organizing architecture for WSNs was proposed in~\cite{Subramanian00},
where authors defined components necessary for building such an architecture.
The approach is designed for static WSNs and not applicable for systems with
high dynamics -- for example, wildlife monitoring -- since the network changes
badly and rapidly. Nevertheless, the work is focused on the whole network, while
our approach is node-centric.

Other attempt to combine both algorithmic and architecture self-adaptation was
performed by Diguet et al.~\cite{Diguet11}. In this work authors blur the
boundaries between software and hardware to gain more flexibility. It leads,
however, to ad-hoc implementation with sophisticated self-adaptive
functionality. Contrary, COP concepts allows to build a holistic and easy to
understand design.

The COP model has been implemented in several high-level
languages~\cite{Bardram05,Ghezzi10,Kamina11,Salvaneschi12,Sehic11}, which are
not applicable in WSNs due to resource limitations. We borrow a few of these
concepts, however, and adapt them to the limitations of WSN platforms. Thus, our
layered functions are the representatives of layer-in-class concept by
Salvaneschi et al.~\cite{Salvaneschi12}.

WSNs are considered as raw sources of data, so far. For example, Sehic et
al.~\cite{Sehic11} design a Java-based framework for context-aware applications,
which are using the input data from WSN. Differenly, we put the notion of
contexts down to the resource-constrained node, allowing to implement
context-aware behavior on the device that interacts with the environment
directly.

In the literature there are seen also works towards run-time adaptation of WSNs.
Zimmerling et al.~\cite{Zimmerling12}, for example, focus on an adaptation of
the parameters of MAC protocol depending on link, topology, and traffic
dynamics. The proposed Java-based framework –- pTunes –- calculates the
parameters, performs decision and sens the parameters to the nodes. In that
case, to be adaptive, nodes are constantly connected to the base station.
Another adaptive routing protocol is designed by Bourdenas et
al.~\cite{Bourdenas11}, where authors proposed forecasting approach to predict
the parameters and conditions of the network.

These efforts are solving different aspects of the same problem -- effective
adaptivity of WSNs -- that we address in this work. The complexity of the
applications we target arises from the different environmental dimensions the
software should adapt to. We also overcome resource limitations to bring the
adaptation on the low-level programming platforms.
