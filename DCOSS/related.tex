\section{Related Work}\label{sec:related}

Over decade many contributions are addressing to the self-adaptivity of WSNs in
both hardware and software perspectives. Unlike in our work, they are mostly
focusing on a few environmental dimensions, while the adaptation rarely occurs
individually. Moreover, the solutions address the actual problem-specific
adaptation logic, while our approach provides abstractions to design
self-adaptive software regardless the problem.

The self-organizing architecture for WSNs was proposed in~\cite{Subramanian00},
where authors defined components necessary for building such an architecture.
The approach is designed for static WSNs and not applicable for systems with
high dynamics -- for example, wildlife monitoring -- since the network changes
badly and rapidly. Nevertheless, the work is focused on the whole network, while
our approach is node-centric.

Other attempt to combine both algorithmic and architecture self-adaptation was
performed by Diguet et al.~\cite{Diguet11}. In his work authors blur the
boundaries between software and hardware to gain more flexibility. It leads,
however, to ad-hoc implementation with sophisticated self-adaptive
functionality. Contrary, COP concepts allows to build a holistic and easy to
understand design.

The COP model has been implemented in several high-level languages~\cite{Bardram05,Ghezzi10,Kamina11,Salvaneschi12,Sehic11}, which are not applicable in WSNs due to resource limitations.

design-time support for self-adaptive software 

The work in the area of self-adaptive WSNs contains contributions in different perspectives -- e.g. architecture~\cite{}, hardware, software  of adaptivity. Unlike in our work, they are mostly focusing at a few environmental dimensions.