\begin{abstract}

Wireless sensor networks (WSNs) take a significant role to gather data from, and
possibly take actions on the real world. Therefore they are intimately tied to
the environment they operate in. Because of unpredicted dynamics of the
environment, there are many situations the software for WSNs needs to self-adapt
to. The complexity of such software is high in general and even higher under
limited resources. This work is a first step to provide a design-time support
for developing self-adaptive software for WSN. To achieve this against severe
resource constraints, we borrow notions from context-oriented programming (COP)
and bring them down to low-level software. As we show in this work, along with
bringing COP, our concepts -- when implemented in a language -- make software
components much more organized and decoupled, and, hence, less complex. The
price to be payed is negligible though.

\end{abstract}